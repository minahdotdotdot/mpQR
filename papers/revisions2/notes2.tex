\documentclass{article}
%Packages
\usepackage[utf8]{inputenc}
\usepackage{geometry, graphicx}
\usepackage{enumerate}
\usepackage{amsmath,amssymb,amsfonts,amsthm, bm}
\usepackage{xcolor, ulem} %just for visible comments and edits.
\usepackage[linesnumbered,ruled,vlined]{algorithm2e}
\usepackage[sort&compress, numbers]{natbib}
%\usepackage[toc,page]{appendix}

% New theorems and commands
\newtheorem{theorem}{Theorem}[section]
\newtheorem{lemma}[theorem]{Lemma}
\newtheorem{corollary}[theorem]{Corollary}
\newcommand\mycommfont[1]{\ttfamily\textcolor{blue}{#1}}
\newcommand{\R}{\mathbb{R}}
\newcommand{\F}{\mathbb{F}}
\newcommand{\dd}{\delta}
\newcommand{\tth}{\theta}
\newcommand{\bb}[1]{\mathbf{#1}}
\newcommand{\fl}{\mathrm{fl}}
\SetCommentSty{mycommfont}
\theoremstyle{definition}
\newtheorem{definition}{Definition}[section]

% Document
\title{Notes and random things}
\author{L. Minah Yang, Alyson Fox, and Geoffrey Sanders}
\date{\today}
\begin{document}
\titlepage


\section{Introduction}

%Given a matrix $\bb{A} \in \mathbb{R}^{m \times n}$, we consider performing the so-called {\it QR factorization}, 
%where
%$$
%\bb{A} = \bb{QR},
%\qquad
%\bb{Q} \in \mathbb{R}^{m \times n},
%\qquad
%\bb{R} \in \mathbb{R}^{n \times n},
%$$
%and $\bb{Q}$ is orthogonal, $\bb{Q}^\top \bb{Q} = I$, and  is upper-triangular, $\bb{R}_{ij} = 0$ for $i>j$.


\subsection{Mixed Precision and Modern Hardware}

\subsection{Notation}
%\begin{table}[h]
%\centering
%\begin{tabular}{|c|c|c|}
%\hline
%Symbol(s) & Definition(s) & Suggestion \\
%\hline
%$\fl$ &floating point operations & \\
%${\bb x}$/${\bb A}$ & vectors/matrices &\\ 
%$m/n$ & num rows/columns in $\bb{A}$&  \\
%$\mu$ & mantissa & \\
%$k$ & num flops & \\
%
%$\bb{x}_i$ & $i^{th}$ index of vector $\bb{x}$ & \\
%$s, p, w$ & sum, product, and storage (write) & \\ 
%$\eta$ & exponent bits &  \\
%$\hat{e}_i$ & cardinal vectors& \\
%$i/j$ & row/column index of a matrix or vector & \\
%$u_q$ & unit round-off for precision $\bb{Q}$ & \\
%$\dd_{q}$ &defined only by $|\dd_{q}| < u_q$ & \\
%$\gamma_{q}^{(k)}$ & $\frac{ku_q}{1-ku_q}$ & \\
%$\tth_{q}^{(k)}$ & defined only by $|\tth_{q}^{(k)}|\leq\gamma_q^{(k)}$ &\\
%$\gamma_{p,q}^{(k_p,k_q)}$ & $(1+\gamma_p^{(k_p)})(1+\gamma_q^{(k_q)})-1$ & \\
%$|x|$; $\|{\bf x}\|_2$ & matrix 2-norm & double bars throughout \\
% $I_{m\times n}$ &  $\begin{bmatrix}
%I_{n\times n}\\
%0_{m-n \times n}\end{bmatrix}$ & \\
%$\bb{A}[a:b, c:d]$ &rows $\bb{A}$ to $b$ and columns $c$ to $d$ of matrix $\bb{A}$ & \\
%$\bb{A}[:,c:d]$ & columns $c$ to $d$ of matrix $\bb{A}$ & \\
%\hline
%\end{tabular}
%\caption{Notation discrepancies and suggestions.   TODO: resolve each row, comment out, and replace for an eventual notation summary table.}
%\end{table}
%%TODO: add to notation summary table? A, Q, R


\section{Floating Point Numbers and Error Analysis Tools}
%
%% DONE: replace m and e with $\mu$ and $\eta$ ?
%
%We will be using floating-point operation error analysis tools developed in \cite{Higham2002}.
%Let $\F \subset \R$ denote the space of some floating point number system with base $\beta$, precision $t$, significand/mantissa $\mu$, and exponent range $\eta_{\text{ran}}:=\{\eta_{\text{min}}, \eta_{\text{min}}+1, \cdots, \eta_{\text{max}} \}$.
%Then every element $y$ in $\F$ can be written as 
%\begin{equation}
%y = \pm \mu\times \beta^{\eta-t},
%\label{eqn:FPbasic}
%\end{equation} 
%where $\mu$ is any integer in $[0,\beta^{t}-1]$, and $\eta\in \eta_{\text{ran}}$.
%While operations we use on $\R$ cannot be replicated exactly due to the finite cardinality of $\F$, we can still approximate the accuracy of analogous floating point operations using these error analysis tools in \cite{Higham2002}. \par
%
%\begin{table}[h]
%	\centering
%	\begin{tabular}{||l|c|c|c|c|c|c||} 
%		\hline 
%		Name & $\beta$ & $t$ & \# of exponent bits & $\eta_{\text{min}}$ & $\eta_{\text{max}}$ & u \\ \hline 
%		IEEE754 half & 2 & 11 & 5 & -15 & 16  & {\tt 4.883e-04} \\ \hline 
%		IEEE754 single & 2 & 24 & 8 & -127 & 128  & {\tt 5.960e-08} \\ \hline 
%		IEEE754 double& 2 & 53 & 11 & -1023 & 1024 & {\tt 1.110e-16} \\ \hline 
%	\end{tabular}
%	\caption{IEEE754 formats with $j$ exponent bits range from $1-2^{j-1}$ to $2^{j-1}$.}
%	\label{table:ieee}
%\end{table}
%
%% GEOFF: reordered to introduce $u$
%A short analysis of floating point operations (cf. Theorem 2.2 \citep{Higham2002}) shows that the relative error is 
%controlled by the unit round-off, $u:=\frac{1}{2}\beta^{1-t}$. 
%Table \ref{table:ieee} shows IEEE precision types described by the same parameters as in Equation \ref{eqn:FPbasic}.
%The true value $(x\text{ op }y)$ lies in $\R$ and it is rounded to the nearest floating point number, $\fl(x\text{ op }y)$, admitting a rounding error. 
%Suppose that a single basic floating-point operation yields a relative error, $\dd$, bounded in the following sense,
%\begin{equation}
%\fl(x\text{ op }y) = (1 + \dd)(x\text{ op }y),\quad |\dd|\leq u, \quad \text{op}\in\{+, -, \times, \div\} \label{eqn: singlefpe}
%\end{equation}
%%The true value $(x\text{ op }y)$ lies in $\R$ and it is rounded to the nearest floating point number, $\fl(x\text{ op }y)$, admitting a rounding error. 
%%A short analysis (cf. Theorem 2.2 \citep{Higham2002}) shows that the relative error $|\dd|$ is bounded by the unit round-off, $u:=\frac{1}{2}\beta^{1-t}$. \par
%
%
%% DONE: replace n with $k$, as $n$ is the matrix dimension?
%We use Equation \ref{eqn: singlefpe} as a building block in accumulating errors from $k$ successive floating point operations in product form.
%Lemma \ref{lem:gamma} introduces new notations that simplify round-off error analyses. 
%\begin{lemma}[Lemma 3.1 \cite{Higham2002}]
%	\label{lem:gamma}
%	Let $|\dd_i|<u$ and $\rho_i \in\{-1, +1\}$ for $i = 1 , \cdots, k$, and $ku < 1$. Then, 
%	\begin{equation}
%	\prod_{i=1}^k (1+\dd_i)^{\rho_i} = 1 + \tth^{(k)}
%	\end{equation}
%	where
%	\begin{equation}
%	|\tth^{(k)}|\leq \frac{ku}{1-ku}=:\gamma^{(k)}.
%	\end{equation}
%\end{lemma}
%In other words, $\tth^{(k)}$ represents the accumulation of $k$ successive round-off errors($\dd$'s), and it is bounded by $\gamma^{(k)}$. 
%This notation often provides upper bounds for relative error, and requiring $\gamma^{(k)}<1$ ensures that the error bound is meaningful. 
%While the assumption $ku<\frac{1}{2}$ (which implies $\gamma^{(k)} < 1$) is satisfied by fairly large $k$ in
%single and double precision types, it is a problem for small $k$ in lower precision types.
%Table \ref{table:ieeen} shows the maximum value of $k$ that still guarantees a relative error below $100\%$ ($\gamma^{(k)} < 1$). 
%\begin{table}[h]
%	\centering
%	\begin{tabular}{||c|c|c||} 
%		\hline
%		precision & $u$ &$\tilde{k} = \mathrm{argmax}_k(\gamma^{(k)} \leq 1)$ \\ \hline
%		half & {\tt 4.883e-04} &$512$\\
%		single & {\tt 5.960e-08} &$4194304\approx 4.19\times 10^6$ \\
%		double &  {\tt 1.110e-16} &$2251799813685248 \approx 2.25\times 10^{15}$ \\ \hline
%	\end{tabular}
%	\caption{Upper limits of validity in the $\gamma^{(k)}$ notation.}
%	\label{table:ieeen}
%\end{table}
%% TODO: u is in the other table as well. 
%
%This reflects on two sources of difficulty: 1) Successive operations in lower precision types grow unstable more quickly, and 2) the upper bound given by $\gamma^{(k)}$ becomes suboptimal faster in low precision.
%However, error analysis within the framework given by Lemma \ref{lem:gamma} best allows us to keep the analysis simple.
%We will use it to study variable-precision block QR factorization methods. \par 
%
%
%% DONE: replace $st$ with $w$that does not contain $s$ or $p$?
%In Lemma \ref{lem:mp}, we present modified versions of relations in Lemma 3.3 in \citep{Higham2002}.
%These relations allow us to easily deal with accumulated errors, and aid in writing clear and simpler error analyses.
%The modifications support multiple precision types, whereas \citep{Higham2002} assumes that the same precision is used in all operations. 
%% TODO: should we call this type uniform precision? I called it that later on.
%
%We distinguish between the different precision types using subscripts--- these types include products ($p$), sums ($s$), and storage formats ($w$).
%
%% DONE: should it be \theta_q^{(k)} ? yes!
%\begin{lemma}[Mixed precision version of Lemma 3.3 from \citep{Higham2002}]
%	\label{lem:mp}
%	For any nonnegative integer $k$ and some precision $q$, let $\tth_{q}^{(k)}$ denote a quantity bounded according to $|\tth_q^{(k)}|\leq \frac{k u^{(q)} }{1-ku^{(q)}} =:\gamma_{q}^{(k)}$.
%	The following relations hold for two precisions $s$ and $p$, positive integers, $j_s$,$j_p$, non-negative integers $k_s$ and $k_p$, and $c>0$.
%	%Most of these result from commutativity. 
%	\begin{align}
%	(1+\tth_{p}^{(k_p)})(1+\tth_{p}^{(j_p)})(1+\tth_{s}^{(k_s)})(1+\tth_{s}^{(j_s)})&=(1+\tth_{p}^{(k_p+j_p)})(1+\tth_{s}^{(k_s+j_s)}) \\
%	\frac{(1+\tth_{p}^{(k_p)})(1+\tth_{s}^{(k_s)})}{(1+\tth_{p}^{(j_p)})(1+\tth_{s}^{(j_s)})} &=\left\{\begin{alignedat}{2}
%	(1+\tth_{s}^{(k_s+j_s)})(1+\tth_{p}^{(k_p+j_p)})&,\quad& j_s \leq k_s, j_p \leq k_p\\
%	(1+\tth_{s}^{(k_s+2j_s)})(1+\tth_{p}^{(k_p+j_p)})&,\quad& j_s \leq k_s, j_p > k_p\\
%	(1+\tth_{s}^{(k_s+j_s)})(1+\tth_{p}^{(k_p+2j_p)})&,\quad& j_s > k_s, j_p \leq k_p \\
%	(1+\tth_{s}^{(k_s+2j_s)})(1+\tth_{p}^{(k_p+2j_p)})&,\quad& j_s > k_s, j_p > k_p
%	\end{alignedat}\right.
%	\end{align}
%	Without loss of generality, let $1 \gg u_p \gg u_s>0$.
%	Let $d$, a nonnegative integer, and $r\in[0, \lfloor\frac{u_p}{u_s}\rfloor]$ be numbers that satisfy $k_su_s = d u_p + r u_s$. Alternatively, $d$ can be defined by $d := \lfloor\frac{k_su_s}{u_p}\rfloor$.
%	\begin{align}
%	\gamma_{s}^{(k_s)}\gamma_{p}^{(k_p)} &\leq \gamma_{p}^{(k_p)}, \quad\text{for } k_p u_p \leq \frac{1}{2}  \\
%	\gamma_{s}^{(k_s)}+u_p &\leq \gamma_{p}^{(d+2)} \\
%	\gamma_{p}^{(k_p)} + u_{s} &\leq \gamma_{p}^{(k_p+1)} \quad{\color{blue}\text{(A loose bound)}}\\ 
%	\gamma_{p}^{(k_p)}+\gamma_{s}^{(k_s)}+\gamma_{p}^{(k_p)}\gamma_{s}^{(k_s)} & < \gamma_{p}^{(k_p+ d+ 1)}\label{lem:mp1}
%	\end{align}
%\end{lemma}
%
%% TODO GEOFF and MINAH.   Meet and work out exact wording to fully formalize.   Should the wording be: there exists a \theta bounded by a \gamma (which is a specific value)?
%
%A proof for Equation \ref{lem:mp1} is shown in Appendix \ref{appendix:A}.

\section{Householder QR Backward Error Analysis}
We present an error analysis for the Householder QR factorization where all inner products are done with precision $p$ for products, and precision $s$ for the inner product, and stored in $w$ precision.

\subsection{Householder QR Factorization Algorithm}
%\label{sec: HQRf}
%% DONE define cardinal vector (in the notation section above, perhaps)
%
%The Householder QR factorization uses Householder transformations to zero out elements below the diagonal of a matrix. 
%First, we consider the simpler task of zeroing out all but the first element of a vector, $\bb{x}\in\R^m$.
%\begin{lemma}
%	Given vector $\bb{x}\in\R^{m}$, there exist Householder vector $\bb{v}$ and Householder transformation matrix $\bb{P}_{\bb{v}}$ such that $\bb{P}_{\bb{v}}$ zeroes out $\bb{x}$ below the first element. 
%	\begin{equation}
%	\begin{alignedat}{3} 
%	\sigma =& -\rm{sign}(x_1)\|\bb{x}\|_2, &&\quad  \bb{v} = \bb{x} -\sigma \hat{e_1},\\
%	\beta = & \frac{2}{\bb{v}^{\top}\bb{v}}=-\frac{1}{\sigma\bb{v}_1}, && \quad \bb{P}_{\bb{v}}=  I - \beta \bb{v}\bb{v}^{\top}
%	\end{alignedat}
%	\end{equation}
%	The resulting vector has the same 2-norm as $\bb{x}$ since Householder transformations are orthogonal.
%	\begin{equation}
%		  \bb{P}_{\bb{v}}\bb{x} = \sigma\hat{e_1}
%	\end{equation}
%	In addition, $\bb{P}_{\bb{v}}$ is symmetric and orthogonal ($\bb{P}_{\bb{v}}=\bb{P}_{\bb{v}}^{\top}=\bb{P}_{\bb{v}}^{-1}$), and therefore involutary ($\bb{P}_{\bb{v}}^2=\bb{I}$).
%	\label{lem:hhvec}
%\end{lemma}
%% DONE: change above Lemma to a definition?   Or, add the fact that \bb{P}_v x is zeroed out below the first element 
%% DONE: \bb{P}_v is of the form I-P, which is confusing.
%
%
%% DONE: define $\bb{v}_i$.   I suggest definining it first and having the formal lemma(or definition) describe the general case.
% 
%Given $\bb{A}\in\mathbb{R}^{m\times n}$ and Lemma \ref{lem:hhvec}, a Householder QR factorization is done by repeating the following processes.
%For $i = 1, 2, \cdots, n,$
%\begin{enumerate}[Step 1)]
%	\item Find and store the Householder constant ($\bm{\beta}_i$) and vector $\bb{v}_i$ that zeros out the $i^{\text{th}}$ column beneath the $i^{\text{th}}$ element,
%	\item Apply the corresponding Householder transformation to the appropriate bottom right partition of the matrix,
%	\item Move to the next column,
%\end{enumerate}
%until only an upper triangular matrix remains. 
%
%Consider the following $4$-by-$3$ matrix example adapted from \cite{Higham2002}. 
%Let $\bb{P}_i$ represent the $i^{th}$ Householder transformation of this algorithm. 
%\[A = \left[ \begin{array}{ccc}
%\times & \times & \times \\
%\times & \times & \times \\
%\times & \times & \times \\
%\times & \times & \times
%\end{array}
%\right]\xrightarrow{\bb{P}_1}\left[ \begin{array}{c|cc}
%\times & \times & \times \\ \hline
%0 & \times & \times \\
%0 & \times & \times \\
%0 & \times & \times
%\end{array}
%\right]
%\xrightarrow{\bb{P}_2} \left[
%\begin{array}{cc|c}
%\times & \times & \times \\
%0 & \times & \times \\ \hline
%0 & 0 & \times \\
%0 & 0 & \times 
%\end{array} \right]
%\xrightarrow{\bb{P}_3} \left[ \begin{array}{ccc}
%\times & \times & \times \\
%0 & \times & \times \\
%0 & 0 & \times \\
%0 & 0 & 0 
%\end{array}\right] \] 
%
%Since the final matrix $ \bb{P}_3\bb{P}_2\bb{P}_1\bb{A}$ is upper-triangular, this is the $\bb{R}$ factor of the QR decomposition.
%Set $\bb{Q}^{\top}:=\bb{P}_3\bb{P}_2\bb{P}_1$. 
%Then we can formulate  $\bb{Q}$ via: 
%$$
%\bb{Q} = (\bb{P}_3\bb{P}_2\bb{P}_1)^{\top} = \bb{P}_1^{\top}\bb{P}_2^{\top}\bb{P}_3^{\top} = \bb{P}_1\bb{P}_2\bb{P}_3,
%$$
%where the last equality results from the symmetric property of $\bb{P}_i$'s. 
%In addition, this is orthogonal because $\bb{Q}^{\top}=\bb{P}_3\bb{P}_2\bb{P}_1 =  \bb{P}_3^{\top}\bb{P}_2^{\top}\bb{P}_1^{\top} =  \bb{P}_3^{-1}\bb{P}_2^{-1}\bb{P}_1^{-1}=(\bb{P}_1\bb{P}_2\bb{P}_3)^{-1}=\bb{Q}^{-1}$, where the third equality results from the orthogonal property of $\bb{P}_i$'s.
%
%Returning to the general case, we have: 
%
%\begin{equation}
%\bb{Q} = \bb{P}_1 \cdots \bb{P}_n,\quad \text{and} \quad \bb{R} = \bb{Q}^{\top}\bb{A} = \bb{P}_n\cdots \bb{P}_1\bb{A}.
%\end{equation}
%% TODO: should \bb{P}_i be \bb{P_i} so its not confused as the element of a vector notation?
\subsection{Inner product error}
%As seen from the previous section, the inner product is a building block of the Householder QR method.
%More generally, it is used widely in most linear algebra tools.
%Thus, we will generalize classic round-off error analysis of inner products to multiple precision. \par
%Specifically, we consider performing an inner product with different floating point precision assigned to operations multiplication and addition.
%This is designed to provide a more accurate rounding error analysis of mixed precision floating point operations in recent GPU technologies such as NVIDIA's TensorCore. 
%Currently, TensorCore computes the inner product of vectors stored in half-precision by employing full precision multiplications and a single-precision accumulator. 
%%TODO: citation?
%As the majority of rounding errors from computing inner products occur during summation, this immensely reduces the error in comparison to using only half-precision operations.
%This increase in accuracy combined with speedy performance motivates us to: 1) study how to best utilize mixed-precision arithmetic in algorithms, and 2) to develop error analysis for mixed-precision algorithms to better understand them.
%%TODO: is this a big/vague/unsupported claim?
%
%\begin{lemma}
%	\label{lem:ip_a}
%	Let $w$, $p$, and $s$ each represent floating-point precisions for storage, product, and summation, where the varying precisions are defined by their unit round-off values denoted by $u_w$, $u_p$, and $u_s$.
%	Let $\bb{x},\bb{y}\in \mathbb{F}_w^{m}$ be two arbitrary $m$-length vectors stored in $w$ precision.
%	If an inner product performs multiplications in precision $p$, and addition of the products using precision $s$, then,
%	\begin{equation}
%	\fl(\bb{x}^{\top}\bb{y}) = (\bb{x}+\bb{\Delta x}) \bb{y} = \bb{x}(\bb{y}+\bb{\Delta y}),
%	\end{equation}
%	where $|\bb{\Delta x}|\leq \gamma_{p,s}^{(1,m-1)}|\bb{x}|$, $|\bb{\Delta y}|\leq \gamma_{p,s}^{(1,m-1)}|\bb{y}|$ componentwise, and $$\gamma_{p,s}^{(1,m-1)} := (1+u_p)(1+\gamma_s^{(m-1)})-1.$$
%	If we further assume that this result is then stored in precision $w$, and $u_w=u_p$, then $|\bb{\Delta x}|\leq \gamma_w^{(d+2)}|\bb{x}|$ and $|\bb{\Delta y}|\leq \gamma_w^{(d+2)}|\bb{y}|$ where $d:=\lfloor\frac{(m-1)u_s}{u_w}\rfloor$.
%\end{lemma}
%
%% DONE fix boldface for vectors,  use a single letter for st
%% TODO $\|\|_2$ for norms, What if x and or why have some zeros, or very small values?
%% TODO is it really component-wise or in the infinity norm?   
%
%
%\begin{lemma}
%	\label{lem:ip_b}
%	Let $w$ and $s$ each represent floating-point precisions for storage and summation, where the unit round-off values for each precision are denoted by $u_w$ and $u_s$. 
%	Futhermore, assume $1\gg u_w \gg u_s>0$, and that for any two arbitary numbers $x$ and $y$ in $\mathbb{F}_w$, their product  $xy$ is in $\mathbb{F}_s$.
%	Let $\bb{x},\bb{y}\in \mathbb{F}_w^{n}$ be two arbitrary $n$-length vectors stored in $w$ precision.
%	If an inner product performs multiplications in full precision, and addition of the products using precision $s$, then,
%	\begin{equation}
%	\fl(\bb{x}^{\top}\bb{y}) = (\bb{x}+\bb{\Delta x}) \bb{y} = \bb{x}(\bb{y}+\bb{\Delta y}),
%	\end{equation}
%	where $|\Delta x|\leq \gamma_w^{(d+1)}|x|$, $|\Delta y|\leq \gamma_w^{(d+1)}|y|$ componentwise, and $d:=\lfloor\frac{(n-1)u_s}{u_w}\rfloor$.
%\end{lemma}
%
%Proofs for Lemmas \ref{lem:ip_a} and \ref{lem:ip_b} are shown in Appendix \ref{appendix:A}.
%The analyses for these two lemmas differ only in the type of mixed-precision arithmetic performed within the inner product subroutine.
%For the rest of this paper, we will refer to the forward error bound for the inner product as $\gamma_w^{d+z}$ for $z=1,2$ to generalize the analysis for varying assumptions.
%%TODO: Is this unclear? We have d+1 for the first lemma, and d+2 for the second lemma, so I just want to generalize that to z. d is also a small integer. 
%This simplification allows us to use the same analysis for the remaining steps of the Householder QR algorithm since inner products are the only computation that use mixed-precision arithmetic. % TODO: subject verb?!

\subsection{Calculation and normalization of Householder Vector}
%An efficient algorithm for calculating $\bb{v}$ is shown in Algorithm \ref{algo:hh_v1}.
%\begin{algorithm}
%	\DontPrintSemicolon % Some LaTeX compilers require you to use \dontprintsemicolon instead
%	\KwIn{$\bb{x}\in\R^m$}
%	\KwOut{$\bb{v}\in\R^m$, and $\sigma, \beta\in\R$ such that $(I-\beta \bb{v} \bb{v}^{\top})\bb{x} = \pm \|\bb{x}\|_2 \hat{e_1} = \sigma\hat{e_1}$ }
%	\tcc{We choose the sign of sigma to avoid cancellation of $\bb{x}_1$ (As is the standard in LAPACK, LINPACK packages \cite{Higham2002}). This makes $\beta>0$.}
%	$\bb{v}\gets \bb{x}$\\
%	$\sigma \gets -\rm{sign}(\bb{x}_1)\|\bb{x}\|_2$\\
%	$\bb{v}_1 \gets \bb{v}_1-\sigma$\\
%	$\beta \gets -\frac{1}{\sigma \bb{v}_1}$\\
%	\Return $\beta$, $\bb{v}$
%	\caption{Given a vector $\bb{x}\in\R^n$, return a Householder vector $\bb{v}$ and a Householder constant $\beta$ such that $(I-\beta \bb{v}\bb{v}^{\top})\bb{x} \in \mathrm{span}(\hat{e_1})$.}
%	\label{algo:hh_v1}
%\end{algorithm}
%
%The above algorithm leaves $\bb{v}$ unnormalized, but it is often normalized via the various methods and reasons listed below:
%\begin{itemize}
%	\item Set $\bb{v}_1$ to $1$ for efficient storage of many Householder vectors.
%	\item Set the 2-norm of $\bb{v}$ to $\sqrt{2}$ to always have $\beta=1$.
%	\item Set the 2-norm of $\bb{v}$ to $1$ to prevent extremely large values, and to always have $\beta=2$.
%\end{itemize}
%
%%TODO: Should the itemize be in a table instead?
%The first normalizing method adds an extra rounding error to $\beta$ and $\bb{v}$ each, whereas the remaining methods incur no rounding error in forming $\beta$ since $1$ and $2$ can be represented exactly.
%The LINPACK implementation of the Householder QR factorization uses {\color{blue}CHECK!} the first method of normalizing via setting $\bb{v}_1$ to $1$. 
%Algorithm \ref{algo:hh_v2} shows how this convention could be carried out. 
%The error analysis in the subsequent section assumes that there may exist errors in both $\beta$ and $\bb{v}$ to get the worse-case scenario and to be consistent with the LINPACK implementation. 
%
%%TODO: add short sentence describing LINPACK and citation
%\begin{algorithm}
%	\DontPrintSemicolon % Some LaTeX compilers require you to use \dontprintsemicolon instead
%	\KwIn{$\bb{x}\in\R^m$}
%	\KwOut{$\bb{v}\in\R^m$, and $\sigma, \beta\in\R$ such that $(I-\beta \bb{v}\bb{v}^{\top})\bb{x} = \pm \|\bb{x}\|_2 \hat{e_1} = \sigma\hat{e_1}$ }
%	\tcc{We choose the sign of sigma to avoid cancellation of $\bb{x}_1$ (As is the standard in LAPACK, LINPACK packages \cite{Higham2002}). This makes $\beta>0$.}
%	$\bb{v}\gets \bb{x}$\\
%	$\sigma \gets -\rm{sign}(\bb{x}_1)\|\bb{x}\|_2$\\
%	$\bb{v}_1 \gets \bb{x}_1-\sigma$ \tcp*{This is referred to as $\bb{\tilde{v}}_1$ later on.} 
%	$\beta \gets -\frac{\bb{v}_1}{\sigma}$\\
%	$\bb{v} \gets \frac{1}{\bb{v}_1}\bb{v}$\\
%	\Return $\beta$, $\bb{v}$, $\sigma$
%	\caption{$\beta$, $\bb{v}$, $\sigma = \mathrm{hh\_vec}(\bb{x})$. Given a vector $\bb{x}\in\R^n$, return the Householder vector $\bb{v}$, a Householder constant $\beta$, and $\sigma$ such that $(I-\beta \bb{v}\bb{v}^{\top})\bb{x} =\sigma(\hat{e_1})$, and $\bb{v}_1=1$.}
%	\label{algo:hh_v2}
%\end{algorithm}
%
%\subsubsection{Error analysis for $\bb{v}$}
%In this section, we show how to bound the error when employing the mixed precision dot product procedure for Algorithm \ref{algo:hh_v2}.
%To do so, we start with the 2-norm error and build from there.
%\par
%
%\begin{lemma}[2-norm error]
%	\label{lem:2norm_a}
%	Let $p$, and $s$ each represent floating-point precisions for storage, product, and summation, where the varying precisions are defined by their unit round-off values denoted by $u_w$, $u_p$, and $u_s$, and we can assume $1\gg u_w \gg u_p,u_s$. 
%	Let $\bb{x}\in \mathbb{F}_w^{m}$ be an arbitrary $n$-length vector stored in $w$ precision.
%	If an inner product performs multiplications in precision $p$, and addition of the products using precision $s$, then,
%	\begin{equation}
%	\fl(\|\bb{x}\|_2)= (1+\tth_w^{(d+z+1)})\|\bb{x}\|_2,
%	\end{equation}
%	where $|\tth_w^{(d+z+1)}|\leq \gamma_w^{(d+z+1)}|\bb{x}|$ for $z\in\{1,2\}$ and $d:=\lfloor\frac{(m-1)u_s}{u_w}\rfloor$.
%\end{lemma} 
%There is no error incurred in evaluating the sign of a number or flipping the sign. 
%Therefore, the error bound for computing $\sigma = -\rm{sign}(\bb{x}_1)\|\bb{x}\|_2$ is exactly the same as that for the 2-norm.
%\begin{equation}
%\label{eqn:sigma}
%\fl(\sigma) = \hat{\sigma} = \rm{fl}(-\rm{sign}(\bb{x}_1)\|\bb{x}\|_2) = \sigma + \Delta \sigma,\quad |\Delta\sigma| \leq \gamma_w^{(d+z+1)}|\sigma|\quad
%\end{equation}
%
%We can now show the error for $\bb{\tilde{v}}_1$ and $\bb{v}_i$ where $i=2 , \cdots, n$. 
%Here $\bb{\tilde{v}}_1$ is still the penultimate value $\bb{v}_1$ held ($\bb{\tilde{v}}_1 = \bb{x}_1-\sigma$).
%Then the round-off errors for $\bb{\tilde{v}}_1$ and $\bb{v}_i$'s are
%\begin{align*}
%\fl(\bb{v}_1)&=\hat{\bb{v}_1} = \bb{\tilde{v}}_1 + \bb{\Delta \tilde{v}}_1 \\
%&= \fl(\bb{x}_1-\hat{\sigma})= (1+\dd_w) (\sigma + \Delta\sigma) = (1+\tth_w^{(d+z+2)})\bb{\tilde{v}}_1\\
%\fl(\bb{v}_i)&=\hat{\bb{v}_i} = \fl(\frac{\bb{x}_i}{\hat{\bb{v}_1}}) = (1+\dd_w)\frac{\bb{x}_i}{\bb{\tilde{v}}_1 + \bb{\Delta \tilde{v}}_1}=(1+\theta_w^{(1+2(d+z+2))})\bb{\tilde{v}}_i.
%\end{align*}
%
%The above equalities are permitted since $\tth$ values are allowed to be flexible within the corresponding $\gamma$ bounds.
%
%\subsubsection{Error analysis for $\beta$}
%Now we show the derivation of round-off error for the Householder constant, $\beta$.
%\begin{align*}
%\hat{\beta} = \fl(-\frac{\hat{\bb{v}_1}}{\fl(\hat{\sigma})}) &=-(1+\dd_w)\frac{\bb{\tilde{v}}_1+\bb{\Delta \tilde{v}}_1}{(\sigma + \Delta\sigma)} \\
%&\leq -(1+\tth_w^{(1)})\frac{ (1+\tth_w^{(d+z+2)})\bb{v}_1}{(1+\tth_w^{(d+z+1)})\sigma}\\
%&\leq (1+\tth_w^{(d+z+3+2(d+z+1))})\beta\\
%&= (1+\tth_w^{(3d+3z+5)})\beta,
%\end{align*}
%where $z=1$ or $z=2$ depending on which mixed-precision inner product procedure was used. 
%
%\subsubsection{Comparison to uniform precision analysis}
%In this paper, uniform precision refers to using the same precision for all floating point operations. 
%We compare the errors for $\hat{\beta}$ and $\hat{\bb{v}}$ computed via the mixed-precision inner products to the errors computed while everything was done in half-precision. 
%Without mixed-precision, the errors would be bounded by
%\begin{equation}
%\tilde{\gamma}^{(k)} := \frac{cku}{1-cku},
%\end{equation}
%and $c$ is a small integer (c.f. Section 19.3 \citet{Higham2002}).
%Let us further assume that the storage precision ($u_{w}$) in the mixed-precision analysis is half-precision. 
%In other words, we can let $u\equiv u_w$, and directly compare $\tilde{\gamma_w}^{(m)}$ and $\gamma_w^{(3d+3z+5)}$.
%The integer $d$ depends on the length of the vector, $m$ and the precisions ($u_w$ and $u_s$), and likely is a small integer.
%For example, if storage is done in half-precision, and summation within the inner product is done in single-precision, $d :=\lfloor\frac{m-1}{8192}\rfloor$.
%Since both $d$ and $z$ are usually small integers, the errors for $\hat{\beta}$ and $\hat{\bb{v}}$ with mixed-precision arithmetic can be approximated by $\gamma_w^{(3d+3z+5)} \approx \tilde{\gamma_w}^{(d+z+1)}$.
%This is an improvement from $\tilde{\gamma_w}^{(m)}$ as$$m \gg \lfloor\frac{m-1}{8192}\rfloor + z + 1.$$

\subsection{Applying a single Householder Transformation}
%Applying a Householder transformation is implemented by a series of inner and outer products, since Householder matrices are rank-1 updates of the identity. 
%This is much less costly than forming $\bb{P}_{\bb{v}}$, then performing matrix-vector or matrix-matrix multiplications.
%For some $\bb{P}_{\bb{v}}=I-\beta \bb{v}\bb{v}^{\top}$, we result in the following computation.
%\begin{equation}
%\bb{P}_{\bb{v}} \bb{x} = (I-\beta \bb{v}\bb{v}^{\top})\bb{x} = \bb{x} - (\beta \bb{v}^{\top}\bb{x})\bb{v}
%\end{equation}
%\subsubsection{Applying $\bb{P}_{\bb{v}}$ to zero out the target column of a matrix}
%Let $\bb{x}\in\R^{m}$ be the target column we wish to zero out beneath the first element.
%Recall that we chose a specific $\bb{v}$ such that $\bb{P}_{\bb{v}}\bb{x} = \sigma \hat{e}_1$. 
%As a result, the only error lies in the first element, $\sigma$, and that is shown in Equation \ref{eqn:sigma}.
%Note that the normalization choice of $\bb{v}$ does not impact the Householder transformation matrix ($\bb{P}_{\bb{v}}$) nor its action on $\bb{x}$, $\bb{P}_{\bb{v}}\bb{x}$.
%
%\subsubsection{Applying $\bb{P}_{\bb{v}}$ to the remaining columns of the matrix}
%Now, let $\bb{x}$ and $\bb{v}$ have no special relationship, as $\bb{v}$ was constructed given some preceding column.
%
%Set $\bb{w}:= \beta \bb{v}^{\top}\bb{x}\bb{v}$.
%Note that $\bb{x}$ is exact, whereas $\bb{v}$ and $\beta$ were still computed. 
%\begin{align*}
%\fl(\bb{\hat{v}}^{\top}\bb{x}) &= (1+\tth_w^{(d+z)})(\bb{v}+\Delta\bb{v})^{\top}\bb{x} \\
%&= (1+\tth_w^{(d+z)})(1+\tth_w^{(1+2(d+z+2))})\bb{v}^{\top}\bb{x}\\
%&= (1+\tth_w^{(3d+3z+5)})\bb{v}^{\top}\bb{x}\\
%\bb{\hat{w}} &=(1+\tth_w^{(2)})(\beta+\Delta\beta)(1+\tth_w^{(3d+3z+5)})\bb{v}^{\top}\bb{x}\bb{w} \\
%&= (1+\tth_w^{(2)})(1+\tth_w^{(3d+3z+5)})\beta(1+\tth_w^{(3d+3z+5)})\bb{v}^{\top}\bb{x}\bb{w}\\
%&= (1+\tth_w^{(6d+6z+12)})\bb{w}\\
%\fl(\bb{x}-\bb{\hat{w}}) &= (1+\dd_w)(1+\tth_w^{6d+6z+12)})\bb{w} \\
%&= (1+\tth_w^{(6d+6z+13)})\bb{P}_{\bb{v}}\bb{x}
%\end{align*}
%
%Constructing $\bb{Q}$ and  both rely on applying Householder transformations in the above two ways: 1) to zero out below the diagonal of a target column, and 2) to update the bottom right submatrix. 
%We now have the tools to formulate the forward error bound on $\hat{\bb{Q}}$ and $\hat{\bb{R}}$ calculated from the Householder QR factorization.

\section{Householder QR}
%The pseudo-algorithm in Section \ref{sec: HQRf} shows each succeeding Householder transformation is applied to a smaller lower right submatrix each time. 
%Consider a thin QR factorization. 
%Then, for $\bb{A}\in\R^{m\times n}$ for $m\geq n$, we have $\bb{A}\in\R^{m\times n}$ and $\bb{R}\in\R^{n\times n}$.
%Everything beneath the diagonal on  is set to zero.
%\begin{align*}
%\hat{\bb{R}}_{ij} & =(1+\tth_w^{(r_{ij})})\bb{R}_{ij} \\
%\hat{\bb{Q}}_{ij} & =(1+\tth_w^{(q_{ij})})\bb{Q}_{ij} \\  
%\end{align*}
%\begin{align*}
%r_{ij} = &\left\{\begin{alignedat}{3}
%&\lfloor\frac{((m-(i-1))u_s}{u_w}\rfloor+z+1)+&&\sum_{k=0}^{i-1}\left(6(\lfloor\frac{(m-k)u_s}{u_w}\rfloor+z)+13 \right) ,\quad &&i= j\\
%& &&\sum_{k=0}^{i-1}\left(6(\lfloor\frac{(m-k)u_s}{u_w}\rfloor+z)+13\right),\quad &&i<j
%\end{alignedat}\right. \\
%q_{ij}=&\left\{\begin{alignedat}{3}
%&&\sum_{k=1}^i\left(6(\lfloor\frac{(m-(k-1))u_s}{u_w}\rfloor+z)+13\right)&,\quad j&&\leq i < n\\
%&&\sum_{k=1}^j\left(6(\lfloor\frac{(m-(k-1))u_s}{u_w}\rfloor+z)+13\right)&,\quad i&&< j < n \\
%13+5(\lfloor\frac{(m-(n-1))u_s}{u_w}\rfloor+z)+&&\sum_{k=1}^{n-1}\left(6(\lfloor\frac{(m-(k-1))u_s}{u_w}\rfloor+z)+13\right)&,\quad j&&\leq i =n \\
%\end{alignedat}\right.
%\end{align*}
%
%For values of $m$, $n$, $u_s$, and $u_w$ such that $d:=\lfloor\frac{mu_s}{u_w}\rfloor=\lfloor\frac{(m-(n-1))u_s}{u_w}\rfloor$, this simplifies. 
%Even when $\lfloor\frac{mu_s}{u_w}\rfloor>\lfloor\frac{(m-(n-1))u_s}{u_w}\rfloor$, the same analysis can be used as an upper bound.
%\begin{align*}
%r_{ij} = &\left\{\begin{alignedat}{2}
%&(6i+1)d+(6i+1)z+ 13i+1,\quad &&i= j\\
%&i\left(6d+6z+13\right),\quad &&i<j
%\end{alignedat}\right. \\
%q_{ij}=&\left\{\begin{alignedat}{2}
%i\left(6d+6z+13\right)&,\quad j&&\leq i < n\\
%j\left(6d+6z+13\right)&,\quad i&&< j < n \\
%(6i+5)d+(6i+5)z+ 13i+13&,\quad j&&\leq i =n \\
%\end{alignedat}\right.
%\end{align*}
%
%We can further approximate to get:
%\begin{align*}
%\hat{\bb{R}} &= \bb{R} +\bb{\Delta R} = (1+\tth_w^{((6n+1)d+(6n+1)z+ 13n+1)}) \bb{R}\\
%\hat{\bb{Q}} &= \bb{Q} +\bb{\Delta Q} = (1+\tth_w^{((6n+5)d+(6n+5)z+ 13n+13)}) \bb{Q}\\
%\end{align*}
%%TODO: Should this be $||$ for componentwise?
%%TODO: should matrices also be boldface??
%
%A backward error for $\bb{A}$ can be given from this.
%We use the mixed-precision inner product as a subroutine for this matrix-matrix multiplication.
%\begin{align*}
%\hat{\bb{A}} &= \fl(\hat{\bb{Q}}\hat{\bb{R}}) = \bb{A} + \bb{\Delta A}\\
%&= (1+\tth_w^{((12n+6)d+(12n+6)z+ 26n+14)})(1+\tth_w^{(d+z)}) \bb{A} \\
%&= (1+\tth_w^{((12n+7)d+(12n+7)z+ 26n+14)})\bb{A}\\
%\left|\tth_w^{((12n+7)d+(12n+7)z+ 26n+14)}\right|&\leq \tilde{\gamma_w}^{(10n(d+z+1))}
%\end{align*}
%
%This is an improvement from $\tilde{\gamma_w}^{(mn)}$, since $m \gg 10(d+z+1)$ in a TSQR setting.

\appendix
\section{Appendix: Proofs of Basic Lemmas}
\label{appendix:A}
\subsection{Lemma \ref{lem:mp} (Equation \ref{lem:mp1})}
\begin{proof}
	We wish to round up to the lower precision, $p$, since $1\gg u_p \gg u_s$.  
	Recall that $d := \left\lfloor k_s u_s  / u_p \right\rfloor$ and $r \leq \left\lfloor u_p  / u_s \right\rfloor$,
	and note
	%%% \begin{equation*}
	$ k_pu_p+k_su_s = (k_p+d)u_p + r u_s \leq (k_p+d+1)u_p$. Then,
	%%% \end{equation*}
	\begin{align*}
	\gamma_{p}^{(k_p)}+\gamma_{s}^{(k_s)}+\gamma_{p}^{(k_p)}\gamma_{s}^{(k_s)} 
&= \frac{k_pu_p}{1-k_pu_p} + \frac{k_su_s}{1-k_su_s} + \frac{k_pu_p}{1-k_pu_p}\frac{k_su_s}{1-k_su_s} \\
	= \frac{k_pu_p+k_su_s-k_pk_su_pu_s}{1-(k_pu_p+k_su_s)+k_pk_su_pu_s} %%% \\
	&\leq \frac{(k_p+d+1)u_p-k_pk_su_pu_s}{1-(k_p+d+1)u_p+k_pk_su_pu_s} \\
	&< \frac{(k_p+d+1)u_p}{1-(k_p+d+1)u_p} = \gamma_{p}^{(k_p+d+1)}
	\end{align*}
\end{proof}

\subsection{Inner Products}
\label{appendix:IP}
\subsubsection{Lemma \ref{lem:ip_a}}
Let $\dd_p$ and $\dd_s$ be rounding error incurred from products and summations.
They are bounded by $|\dd_p| < u_p$ and $|\dd_s| < u_s$, following the notation in \cite{Higham2002}. Let $s_k$ denote the $k^{th}$ partial sum, and let $\hat{s_k}$ denote the floating point representation of the calculated $s_k$.
Then,
\begin{align*}
	\hat{s_1} &= \fl (\bb{x}_1\bb{y}_1) = \bb{x}_1\bb{y}_1(1 + \dd_{p,1}),\\
	\hat{s_2} &= \fl(\hat{s_1} + \bb{x}_2\bb{y}_2), \\
	&= \left[\bb{x}_1\bb{y}_1(1 + \dd_{p,1}) + \bb{x}_2\bb{y}_2(1 + \dd_{p,2})\right](1+\dd_{s,1}),\\
	\hat{s_3} &= \fl(\hat{s_2}+\bb{x}_3\bb{y}_3), \\
	&= \left(\left[\bb{x}_1\bb{y}_1(1 + \dd_{p,1}) + \bb{x}_2\bb{y}_2(1 + \dd_{p,2})\right](1+\dd_{s,1})  + \bb{x}_3\bb{y}_3(1+\dd_{p,3})\right)(1+\dd_{s,2}).
\end{align*}
We can see a pattern emerging. 
The error for a general length $m$ vector dot product is then:
\begin{equation}
\label{eqn:dperr_1}
\hat{s_m} = (\bb{x}_1\bb{y}_1+\bb{x}_2\bb{y}_2)(1+ \dd_{p,1})\prod_{k=1}^{m-1}(1+\dd_{s,k}) + \sum_{i=3}^n \bb{x}_i\bb{y}_i(1+\dd_{p,i})\left(\prod_{k=i-1}^{m-1}(1+\dd_{s,k})\right),
\end{equation}
where each occurrence of $\dd_p$ and $\dd_s$ are distinct, but are still bound by $u_p$ and $u_s$.

Using Lemma \ref{lem:gamma}, we further simplify:
\begin{align*}
\fl(\bb{x}^{\top}\bb{y}) &= \hat{s_m} = (1+\tth_p^{(1)})(1+\tth_s^{(m-1)})\bb{x}^{\top}\bb{y}%%%\\
%%%&
= (\bb{x}+\Delta\bb{x})^{\top}\bb{y} = \bb{x}^{\top}(\bb{y}+\Delta\bb{y})
\end{align*}
Here $\Delta\bb{x}$ and $\Delta\bb{y}$ are vector perturbations.

By using Lemma \ref{lem:mp}, Equation \ref{lem:mp1}, we can bound the perturbations componentwise.
Let $d:=\lfloor\frac{(m-1)u_s}{u_p}\rfloor$ such that $(m-1)u_s = d u_p + r u_s$. Then,
\begin{align*}
|\Delta \bb{x}| &\leq \gamma_p^{(d+2)}|\bb{x}| %%% \\
\qquad \mbox{and} \qquad
|\Delta \bb{y}| %%%&
\leq \gamma_p^{(d+2)}|\bb{y}|.
\end{align*}
Furthermore, these bounds lead to a forward error result as shown in Equation \ref{eqn:ipforward},
\begin{equation}
\label{eqn:ipforward}
|\bb{x}^{\top}\bb{y}-\fl(\bb{x}^{\top}\bb{y})| \leq \gamma_p^{(d+2)}|\bb{x}|^{\top}|\bb{y}|.
\end{equation}
%
%While this result does not guarantee a high relative accuracy when $|\bb{x}^{\top}\bb{y}| \ll |\bb{x}|^{\top}|\bb{y}|$, high relative accuracy is expected in some special cases.
%For example, let $\bb{x}=\bb{y}$.
%Then we have exactly $|\bb{x}^{\top}\bb{x}| = |\bb{x}|^{\top}|\bb{x}|=\|\bb{x}\|_2^2$.
%This leads to
%\begin{equation}
%\left|\frac{\|\bb{x}\|_2^2 - \fl(\|\bb{x}\|_2^2)}{\|\bb{x}\|_2^2}\right| \leq \gamma_p^{(d+2)}
%\end{equation}
%

\subsubsection{Corollary \ref{lem:ip_b}}
This proof follows similarly to the proof for Lemma \ref{lem:ip_a}.
Since no error is incurred in the multiplication portion of the inner products, $\dd_s$ and $\dd_{st}$ are rounding error incurred from summations and storage.
The error for a general $m$-length vector dot product is then
\begin{equation}
\label{eqn:dperr_2}
\hat{s_m} = (\bb{x}_1\bb{y}_1+\bb{x}_2\bb{y}_2)\prod_{k=1}^{m-1}(1+\dd_{s,k}) + \sum_{i=3}^n \bb{x}_i\bb{y}_i\left(\prod_{k=i-1}^{m-1}(1+\dd_{s,k})\right).
\end{equation}
Using Lemma \ref{lem:gamma}, we further simplify, 
\begin{equation*}
\fl(\bb{x}^{\top}\bb{y}) = \hat{s_m} = (1+\tth_s^{(m-1)})\bb{x}^{\top}\bb{y}= (\bb{x}+\Delta\bb{x})^{\top}\bb{y} = \bb{x}^{\top}(\bb{y}+\Delta\bb{y}).
\end{equation*}
Here $\Delta\bb{x}$ and $\Delta\bb{y}$ are vector perturbations.

By using Lemma \ref{lem:mp} equation \ref{lem:mp1}, we can bound the perturbations componentwise.
Let $d:=\lfloor\frac{(m-1)u_s}{u_p}\rfloor$ such that $(m-1)u_s = d u_p + r u_s$. 
\begin{align*}
|\Delta \bb{x}| &\leq \gamma_p^{(d+1)}|\bb{x}| %%%\\
\qquad \mbox{and} \qquad
|\Delta \bb{y}| %%%&
\leq \gamma_p^{(d+1)}|\bb{y}| 
\end{align*}
Furthermore, these bounds lead to a forward error result as shown in Equation \ref{eqn:ipforward2},
\begin{equation}
\label{eqn:ipforward2}
|\bb{x}^{\top}\bb{y}-\fl(\bb{x}^{\top}\bb{y})| \leq \gamma_p^{(d+1)}|\bb{x}|^{\top}|\bb{y}|.
\end{equation}

\subsection{Proof for Mixed-Precision HQR result}
\label{Appendix:HQR}
Here, we show a few results that are necessary for the proof for Theorem~\ref{thm:feHQR}.
Lemma~\ref{lem:19.2} shows normwise results for a single mixed-precision Householder transformation performed on a vector, and Lemma~\ref{lem:19.3} builds on Lemma~\ref{lem:19.2} to show normwise results for multiple mixed-precision Householder transformations on a vector. 
We build column-wise results for HQR based on these lemmas and then compute the matrix norms at the end.
\begin{lemma}
	\label{lem:19.2}
	Let $\bb{x}\in\R^m$ and consider the computation of $\bb{y}=\hat{\bb P_v}\bb{x} = \bb{x}-\hat{\beta}\hat{\bb{v}}\hat{\bb{v}}^{\top}\bb{x}$, where $\hat{\bb{v}}$ has accumulated error shown in Lemma~\ref{lem:HQRv}.
	Then, the computed $\hat{\bb{y}}$ satisfies 
	\begin{equation}
	\hat{\bb y} = (\bb{P}+\bb{\Delta P}) \bb{x},\quad \|\bb{\Delta P}\|_F\leq\gamma_w^{(6d+6z+13)},
	\end{equation}
	where $\bb{P} = \bb{I}-\beta\bb{v}\bb{v}^{\top}$ is a Householder transformation.
\end{lemma}
\begin{proof}
	Recall that the computed $\hat{\bb y}$ accumulates component-wise error shown in Equation~\ref{eqn:applyP}.
	Even though we do not explicitly form $\bb{P}$, forming the normwise error bound for this matrix makes the analysis simple.
	First, recall that any matrix $\bb{A}$ with rank $r$ has the following relations between its 2-norm and Frobenius norm,
	%%%\begin{equation}
	$\|\bb{A}\|_2\leq\|\bb{A}\|_F\leq\sqrt{r}\|\bb{A}\|_2$.
	%%%\end{equation}
	Then, we have 
	\begin{equation}
	\label{eqn:19.2a}
	\|\bb{y}\|_2 = \|\bb{P x}\|_2 \leq \|\bb{P}\|_2 \|\bb{x}\|_2 = \|\bb{x}\|_2,
	\end{equation}
	since $\bb{P}$ is orthogonal and $\|\bb{P}\|_2=1$.
	We now transition from the componentwise error to normwise error for $\bb{\Delta y}$, and write $\tilde{z} = 6d+6z+13$:
	\begin{equation}
	\label{eqn:19.2b}
	\|\bb{\Delta y}\|_2 = \left(\sum_{i=1}^m \bb{\Delta y}_i^2\right)^{1/2} \leq \gamma_w^{(\tilde{z})}\left(\sum_{i=1}^m \bb{y}_i^2\right)^{1/2} =  \gamma_w^{(\tilde{z})}\|\bb{y}\|_2
	\end{equation}
	Combining Equations~\ref{eqn:19.2a} and \ref{eqn:19.2b}, we find
	\begin{equation}
	\frac{\|\bb{\Delta y}\|_2}{\|\bb{x}\|_2} \leq \gamma_w^{(\tilde{z})}. \label{eqn:19.2c}
	\end{equation}
	Now, notice that $\bb{\Delta P}$ is exactly $\frac{1}{\bb{x}^{\top}\bb{x}}\bb{\Delta y}\bb{x}^{\top}$; thus, 
	\begin{align*}
	(\bb{P}+\bb{\Delta P}) \bb{x} &= (\bb{P}+\frac{1}{\bb{x}^{\top}\bb{x}}\bb{\Delta y}\bb{x}^{\top})\bb{x}    %%%\\
	%%%&
=\bb{P}\bb{x}  + \frac{\bb{x}^{\top}\bb{x}}{\bb{x}^{\top}\bb{x}}\bb{\Delta y} = \bb{y} + \bb{\Delta y}
	\end{align*}
	We can compute the Frobenius norm of $\bb{\Delta P}$ by using $\bb{\Delta P}_{ij} = \frac{1}{\|\bb{x}\|_2^2}\bb{\Delta y}_i\bb{x}_j$.
	\begin{align*}
	\|\bb{\Delta P}\|_F %%&= \left(\sum_{i=1}^m\sum_{j=1}^m\Delta \bb{P}_{ij}^2\right)^{1/2} 
	= \left(\sum_{i=1}^m\sum_{j=1}^m\left(\frac{1}{\|\bb{x}\|_2^2}\bb{\Delta y}_i\bb{x}_j\right)^2\right)^{1/2} %%%\\
	%%%&= \left(\frac{1}{\|\bb{x}\|_2^4}\sum_{i=1}^m\sum_{j=1}^m\bb{\Delta y}_i^2\bb{x}_j^2\right)^{1/2} 
	%%%=\frac{1}{\|\bb{x}\|_2^2}\left(\sum_{i=1}^m \bb{\Delta y}_i^2\left( \sum_{j=1}^m\bb{x}_j^2\right)\right)^{1/2}\\
	%%%&=\frac{1}{\|\bb{x}\|_2^2}\left(\|\bb{x}\|_2^2\sum_{i=1}^m\bb{\Delta y}_i^2 \right)^{1/2}
	%%%=  \frac{\|\bb{x}\|_2\|\bb{\Delta y}\|_2}{\|\bb{x}\|_2^2} 
        =  \frac{\|\bb{\Delta y}\|_2}{\|\bb{x}\|_2}
	\end{align*}
	Finally, using Equation~\ref{eqn:19.2c}, we find $\|\bb{\Delta P}\|_F \leq \gamma_w^{(\tilde{z})}$.
\end{proof}

%\begin{lemma}
%	\label{lem:3.7}
%	If $\bb{P}_j + \bb{\Delta P}_j\in\R^{m\times m}$ satisfies $\|\bb{\Delta P}_j\|_F\leq\delta_j\|\bb{P}_j\|_2$ for all $j$, then
%	\begin{equation}
%	\left|\left|\prod_{j=0}^m \left(\bb{P}_j + \bb{\Delta P}_j\right) - \prod_{j=0}^m \bb{P}_j\right|\right|_F\leq \left(\prod_{j=0}^m(1+\delta_j)-1\right)\prod_{j=0}^m\|\bb{P}_j\|_2
%	\end{equation}
%\end{lemma}

\begin{lemma}
	\label{lem:19.3}
	Consider applying a sequence of transformations in the set $\{\bb{P}_j\}_{j=1}^r\subset\R^{m\times m}$ to $\bb{x}\in\R^m$, where $\bb{P}_j$'s are all Householder transformations and where we will assume that $r\gamma_w^{(\tilde{z})}<\frac{1}{2}.$ 
	Let $\bb{y} = \bb{P}_r\bb{P}_{r-1}\cdots\bb{P}_1\bb{x} = \bb{Q}^{\top}\bb{x}$.
	Then, $\hat{\bb{y}} = (\bb{Q}+\bb{\Delta Q})^{\top}\bb{x}$, where 
	\begin{equation}
	\|\bb{\Delta Q}\|_F \leq r\gamma_w^{(\tilde{z})},\quad  \|\bb{\Delta y}\|_2 \leq r \gamma_w^{(\tilde{z})} \|\bb{y}\|_2.\label{eqn:19.3}
	\end{equation}
	In addition, if we let $\hat{\bb{y}} =\bb{Q}^{\top}(\bb{x} + \bb{\Delta x})$, then 
	\begin{equation}
	\|\bb{\Delta x}\|_2 \leq r \gamma_w^{(\tilde{z})} \|\bb{x}\|_2.\label{eqn:19.3c}
	\end{equation}
\end{lemma}

\begin{proof}
	As was for the proof for Lemma~\ref{lem:19.2}, we know $\bb{\Delta Q}^{\top} = \frac{1}{\|\bb{x}\|_2^2}\bb{\Delta y}\bb{x}^{\top}$.
	Recall that the HQR factorization applies a series of Householder transformations on $\bb{A}$ to form $\bb{R}$, and applies the same series of Householder transformations in reverse order to $\bb{I}$ to form $\bb{Q}$.
	Therefore, it is appropriate to assume that $\bb{x}$ is exact in this proof, and we form a forward bound on $\hat{\bb{y}}$.
	However, we can still easily switch between forward and backward errors in the following way:
	\begin{align*}
	\hat{\bb{y}} &= \bb{y} + \bb{\Delta y} = \bb{Q}^{\top} (\bb{x}+\bb{\Delta x}) = (\bb{Q}+\bb{\Delta Q})^{\top} \bb{x},\\
	\bb{\Delta y} &= \bb{\Delta Q}^{\top} \bb{x} = \bb{Q}^{\top}\bb{\Delta x}.
	\end{align*}
	In addition, we can switch between $\bb{\Delta y}$ and $\bb{\Delta x}$ by using the fact that $\|\bb{Q}\|_2 = 1$.
	%So another way of formulating $\bb{\Delta Q}$ is given by $\frac{1}{\|\bb{x}\|_2^2}(\bb{Q}\bb{\Delta x})\bb{x}^{\top}$, where $\bb{\Delta x}$ can be understood as the backward error.\par	
	\paragraph{Error bound for $\|\bb{\Delta y}\|_2$}
	We will first find $\|\bb{\Delta Q}\|_2$, where this is NOT the forward error from forming $\bb{Q}$ with Householder transformations, but rather a backward error in accumulating Householder transformations. 
	From Lemma~\ref{lem:19.2}, we have$ \|\bb{\Delta P}\|_F\leq\gamma_w^{(\tilde{z})} = \gamma_w^{(\tilde{z})} \|\bb{P}\|_2$ for any Householder transformation $\bb{P}\in\R^{m\times m}$, where $\tilde{z} = 6d+6z+13$ and $d=\lfloor\frac{(m-1)u_s}{u_w}\rfloor$, $z\in\{1,2\}$.
	Therefore, this applies to the sequence of $\bb{P}_i$'s that form $\bb{Q}$ as well.
	
	We will now use Lemma 3.7 from \cite{Higham2002} to bound $\bb{\Delta Q}$:
	\begin{align*}
	\bb{\Delta Q}^{\top}& = \left(\hat{\bb{Q}} - \bb{Q}\right)^{\top}= \prod_{i=r}^{1}\left(\bb{P}_i +\bb{\Delta P}_i\right) - \prod_{i=r}^{1}\bb{P}_i,\\
	 \|\bb{\Delta Q}\|_F = \|\bb{\Delta Q}^{\top}\|_F  &= \left|\left| \prod_{i=r}^{1}\left(\bb{P}_i +\bb{\Delta P}_i\right) - \prod_{i=r}^{1}\bb{P}_i \right|\right|_F,\\
	&\leq \left(\prod_{i=r}^1(1+\gamma_w^{(\tilde{z})})-1\right)\prod_{i=r}^1\|\bb{P}_i\|_2 = \prod_{i=r}^1(1+\gamma_w^{(\tilde{z})})-1.
	\end{align*}
	The last equality results from the orthogonality of Householder matrices.\par
	
	Consider the constant, $(1+\gamma_w^{(\tilde{z})})^r-1$.
	From the very last rule in Lemma~\ref{lem:up}, we can generalize the following:
	\begin{equation*}
	(1+\gamma_w^{(\tilde{z})})^r = (1+\gamma_w^{(\tilde{z})})^{r-2}(1+\gamma_w^{(\tilde{z})})(1+\gamma_w^{(\tilde{z})}) \leq  (1+\gamma_w^{(\tilde{z})})^{r-2}(1+\gamma_w^{(2\tilde{z})}) \leq \cdots \leq (1+\gamma_w^{(r\tilde{z})}).
	\end{equation*}
	So, our quantity of interest can be bound by $(1+\gamma_w^{(\tilde{z})})^r-1 \leq \gamma_w^{(r\tilde{z})}$.
	
	Now we will use the following equivalent algebraic inequalities to get the final result.
	\begin{equation}
	0<a<b<1 \Leftrightarrow 1-a > 1-b \Leftrightarrow \frac{1}{1-a} <\frac{1}{1-b} \Leftrightarrow \frac{a}{1-a} < \frac{b}{1-b}
	\label{eqn:algebra}
	\end{equation}
	In addition, we assume $r\gamma_w^{(\tilde{z})}< \frac{1}{2}$, such that 
	\begin{align*}
	(1+\gamma_w^{(\tilde{z})})^r-1 &\leq \gamma_w^{(r\tilde{z})} = \frac{r\tilde{z}u_w}{1-r\tilde{z}u_w}\qquad\text{(by definition)}\\
	&\leq \frac{r\gamma_w^{(\tilde{z})}}{1-r\gamma_w^{(\tilde{z})}},\text{ since } r\tilde{z}u_w < r\gamma_w^{(\tilde{z})}\qquad\text{(by Equation \ref{eqn:algebra})}\\
	&\leq 2 r \gamma_w^{(\tilde{z})}\qquad\text{(since $r\gamma_w^{(\tilde{z})}< \frac{1}{2}$ implies  $\frac{1}{1-r\gamma_w^{(\tilde{z})}} < 2$)}\\
	&= r\tilde{\gamma}_w^{(\tilde{z})},
	\end{align*}
	where $\tilde{\gamma^{(m)}}:= \frac{cmu}{1-cmu}$ for some small integer, $c$.
	If we had started with $(1+\tilde{\gamma}_w^{(\tilde{z})})^r-1$, we can still find $r\tilde{\gamma}_w^{(\tilde{z})}$ assuming that $2c$ is still a small integer. 
	In conclusion, we have 
	$$
	(1+\gamma_w^{(\tilde{z})})^r-1 \leq r\tilde{\gamma}_w^{(\tilde{z})},
	$$
	which results in the bound for $\bb{\Delta Q}$ as shown in Equation~\ref{eqn:19.3}, $\|\bb{\Delta Q}\|_2 \leq \|\bb{\Delta Q}\|_F \leq r\gamma_w^{(\tilde{z})}$.

	Next, we bound $\|\bb{\Delta y}\|_2 = \|\bb{\Delta Q x}\|_2 \leq \|\bb{\Delta Q}\|_2 \|\bb{x}\|_2 \leq  r\tilde{\gamma}_w^{(\tilde{z})}\|\bb{x}\|_2$.
	
	\paragraph{Bound for $\|\bb{\Delta x}\|_2$}
	We use the above result,
	\begin{equation}
	\|\bb{\Delta x}\|_2 = \|\bb{Q \Delta y}\|_2\| \leq \|\bb{Q}\|_2\|\bb{\Delta y}\|_2 = \|\bb{\Delta y}\|_2 \leq  r\tilde{\gamma}_w^{(\tilde{z})}\|\bb{x}\|_2.
	\end{equation}
	While $r\gamma^{(k)} = r\frac{ku}{1-ku} < \frac{rku}{1-rku} =\gamma^{(rk)}$ holds true when $r>0$ and $rku< 1$ are satisfied, the strict inequality implies that $r\gamma^{(k)}$ is a tighter bound than $\gamma^{(rk)}$.
	However, $\gamma^{(rk)}$ is easier to work with using the rules in Lemma~\ref{lem:gamma}.
\end{proof}

\subsubsection{Proof for Theorem \ref{thm:feHQR}}
First, we use Lemma~\ref{lem:19.3} directly on columns of $\bb{A}$ and $\bb{I}_{m\times n}$ to get a result for columns of $\hat{\bb{R}}$ and $\hat{\bb{Q}}$.
%Figure~\ref{fig:QRerr} shows that each elements of $\hat{\bb{Q}}$ and $\hat{\bb{R}}$ each go through different numbers of Householder transformations and Householder transformations for different lengths of vectors as well. 
We will use the maximum number of transformations and the length of the longest vector on to which we perform a Householder transformation, that is, $n$ transformations of vectors of length $m$. 
%Also, note that $n\gamma^{(k)} = \frac{nku}{1-ku}<\frac{nku}{1-nku} = \gamma^{(nk)}$ as long as $nku <1$.
For $j$ in $\{1, \cdots, n\}$, the $j^{th}$ column of $\bb{R}$ and $\bb{Q}$ are the results of $j$ Householder transformations on $\bb{A}$ and $\bb{I}$:
\begin{align}
\|\bb{\Delta Q}[:,j]\|_2 &\leq j\tilde{\gamma}_w^{(\tilde{z})}\|\hat{e}_{j}\|_2 < \tilde{\gamma}_w^{(j\tilde{z})}, \\
\|\bb{\Delta R}[:,j]\|_2 &\leq j\tilde{\gamma}_w^{(\tilde{z})}\|\bb{A}[:,j]\|_2 < \tilde{\gamma}_w^{(j\tilde{z})}\|\bb{A}[:,j]\|_2.
\end{align}

Finally, we relate columnwise 2-norms to matrix Frobenius norms.
%\begin{lemma}
%	Let $\bb{A}=[\bb{c}_1 \cdots \bb{c}_n]\in\R^{m\times n}$, where $\bb{c_i}\in\R^m$ for $i=1, \cdots, n$. 
%	If $\max_{i \in\{1, \cdots, n\}}\|\bb{c}_i\| = \epsilon$, then
%	\begin{equation}
%	\|\bb{A}\|_F \leq \sqrt{n}\epsilon
%	\end{equation}
%\end{lemma}
%\begin{proof}
%	The Frobenius norm is exactly the 2-norm for vectors. 
%	\begin{equation*}
%	\|\bb{A}\|_F = \left(\sum_{i=1}^n \sum_{j=1}^m \bb{A}_{ij}^2\right)^{1/2} = \left(\sum_{i=1}^n\|\bb{c}_i\|_2^2\right)^{1/2}\leq \left(\sum_{i=1}^n \epsilon^2\right)^{1/2} = \sqrt{n}\epsilon
%	\end{equation*}
%\end{proof}
It is straightforward to see the result for the $\bb{Q}$ factor,
\begin{equation}
\|\bb{\Delta Q}\|_F = \left(\sum_{j=1}^n \|\bb{\Delta Q}[:,j]\|_2^2\right)^{1/2} \leq \left(\sum_{j=1}^n (j\tilde{\gamma}_w^{(\tilde{z})})^2 \|\hat{e}_j\|_2^2\right)^{1/2} \leq n^{3/2}\tilde{\gamma}_w^{(\tilde{z})}.
\end{equation}
Note that we bound $\sum_{j=1}^n j^2$ by $n^3$, but the summation is actually exactly $\frac{n(n+1)(2n+1)}{6}$. 
Therefore, a tighter bound would replace $n^{3/2}$ with $\left(\frac{n(n+1)(2n+1)}{6}\right)^{1/2}$.\par 
We can bound the $\bb{R}$ factor in a similar way,
\begin{equation}
\|\bb{\Delta R}\|_F = \left(\sum_{j=1}^n \|\bb{\Delta R}[:,j]\|_2^2\right)^{1/2} \leq \left(\sum_{j=1}^n (j\tilde{\gamma}_w^{(\tilde{z})})^2 \|\bb{A}[:,j]\|_2^2\right)^{1/2} \leq n\tilde{\gamma}_w^{(\tilde{z})} \|\bb{A}\|_F.
\end{equation}

Obtaining the backward error from the HQR factorization,
$$\bb{\Delta A}=\bb{A}-\hat{\bb{Q}}\hat{\bb{R}}
= \bb{A}-\bb{Q}\hat{\bb{R}} + \bb{Q}\hat{\bb{R}} - \hat{\bb{Q}}\hat{\bb{R}}
= \bb{Q \Delta R} + \bb{\Delta Q} \hat{\bb{R}}.$$
	
A columnwise result for $\hat{\bb{A}}$ is shown by
\begin{align*}
\|\bb{\Delta A}[:,j]\|_2 & = \|(\bb{Q \Delta R} + \bb{\Delta Q} \hat{\bb{R}} )[:,j]\|_2,\\
&\leq \|\bb{Q \Delta R}[:,j]\|_2  + \|\bb{\Delta Q}\hat{\bb{R}}[:,j]\|_2,\\
&\leq \|\bb{\Delta R}[:,j]\|_2 + \|\bb{\Delta Q}\|_2\|\hat{\bb{R}}[:,j]\|_2,\\
&\leq \|\bb{\Delta R}[:,j]\|_2 + \|\bb{\Delta Q}\|_F\|(\bb{R} +\bb{\Delta R})[:,j]\|_2,\\
&\leq j\tilde{\gamma}_w^{(\tilde{z})}\|\bb{A}[:,j]\|_2 + n^{3/2}\tilde{\gamma}_w^{(\tilde{z})} \|(\bb{Q}^{\top}\bb{A} +\bb{\Delta R})[:,j]\|_2,\\
&\leq j\tilde{\gamma}_w^{(\tilde{z})}\|\bb{A}[:,j]\|_2 + n^{3/2}\tilde{\gamma}_w^{(\tilde{z})} \left(\|\bb{A}[:,j]\|_2 +\|\bb{\Delta R}[:,j]\|_2\right),\\
&\leq \left(j\tilde{\gamma}_w^{(\tilde{z})} + n^{3/2}\tilde{\gamma}_w^{(\tilde{z})}  (1+j\tilde{\gamma}_w^{(\tilde{z})})\right)\|\bb{A}[:,j]\|_2,\\
&=n^{3/2}\tilde{\gamma}_w^{(\tilde{z})}  \|\bb{A}[:,j]\|_2,
\end{align*}
where we assume $n\tilde{\gamma}_w^{(\tilde{z})}\ll 1$ and where the last equality sweeps all non-leading order terms into the arbitrary constant $c$ within the definition of $\tilde{\gamma}$,
\begin{equation}
\|\bb{\Delta A}\|_F = \left(\sum_{j=1}^n \|\bb{\Delta A}[:,j]\|_2^2\right)^{1/2} \leq \left(\sum_{j=1}^n (n^{3/2}\tilde{\gamma}_w^{(\tilde{z})})^2 \|\bb{A}[:,j]\|_2^2\right)^{1/2} \leq n^{3/2}\tilde{\gamma}_w^{(\tilde{z})} \|\bb{A}\|_F.
\end{equation}



\clearpage
\bibliographystyle{plainnat}
\bibliography{report.bib}
	


\end{document}
