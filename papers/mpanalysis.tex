In this section, we consider three different mixed precision settings for the QR factorization, all of which take in a matrix $\bb{A}$ stored in low precision and return $\bb{Q},\bb{R}$ both represented in low precision. 
First, we consider a trivial mixed precision setting where HQR, BQR, and TSQR are computed in high precision after casting up the input matrix at the beginning, and casting down the resulting high precision factors to low precision. 
Then in \cref{sec:mp-3}, we modify BQR and TSQR to utilize level-3 BLAS operations and TensorCore bFMAs for the matrix product subroutines. 
Finally, we impose \cref{assump:mp} in \cref{sec:mp-2} to see how a mixed precision inner product impacts HQR, BQR, and TSQR when applied in level-2 BLAS operations.

\paragraph{Backward error of casting down vectors} First, consider 
%casting down a scalar $x\in\F_h$ to $\F_l$.
%Without overflow or underflow, this results in \[\text{\tt castdown}_{l}(x) = x(1+\dd^{(l)}),\;\; |\dd^{(l)}| < u^{(l)},\]
%and accrues a single rounding error in the lower precision.
%Now, let us consider 
casting down a vector  $\bb{x}\in\F_h^{(m)}$.
The componentwise forward error is, \[\text{\tt castdown}_{l}(\bb{x}) = \bb{x} + \Delta {\bb{x}},\;\; |\Delta\bb{x}| < u^{(l)}|\bb{x}|.\]
We use this to represent the backward error of a casting down a vector with a linear transformation, $\bb{I}^{(l)}:=\bb{I} +\bb{E}\in\R^{m\times m}$, a diagonal perturbation of the identity.
%That is, 
%For some vector $\bb{x}\in\F_h$, the cast down operation yields
We write,
\begin{equation}
\bb{x}^{(l)} := \text{\tt castdown}(\bb{x}^{(h)}) = \bb{I}^{(l)}\bb{x}^{(h)} = (\bb{I}+\bb{E})\bb{x}^{(h)} = \bb{x}^{(h)}+\Delta \bb{x},
\end{equation}
where $|\Delta \bb{x}| \leq u^{(l)} |\bb{x}^{(h)}|$ and  $\|\Delta \bb{x}\|_2 \leq u^{(l)} \|\bb{x}^{(h)}\|_2$.
Thus, $\bb{E} = \Delta \bb{x x}^{\top}/\|\bb{x}\|_2^2$ and we can use the same argument as in \cref{eqn:outer} to form a backward matrix norm bound, 
\begin{equation}
\|\bb{E}\|_F\leq u^{(l)}. \label{eqn:castdown}
\end{equation}

\paragraph{Casting down after HQR in high precision} Let us consider the trivial case of carrying out HQR in high precision and casting down at the very end.
This is useful for the analysis of mixed precision
%applications that require economical storage but have enough memory to carry out HQR in higher precision, or in 
block algorithms as will be shown in \cref{sec:mp-3}.
If the two floating point types $\F_{l}$ and $\F_{h}$ satisfy $\F_{l}\subseteq \F_{h}$
% and for all $x,y\in\F_{l}$, the exact product $xy$ can be represented in $\F_{h}$.
%Some example pairs of $\{\F_{l}, \F_{h}\}$ include $\{\text{fp16}, \text{fp32}\}$, $\{\text{fp32}, \text{fp64}\}$, and $\{\text{fp16}, \text{fp64}\}$.
%Suppose that 
and the matrix to be factorized is stored with low precision numbers, $\bb{A}\in\F_{l}^{m\times n}$, then casting up adds no rounding errors.
Therefore, we can directly apply the analysis that culminated in \cref{thm:feHQR}, and we only consider the columnwise forward error in the $\bb{Q}$ factor.
Then, the $j^{th}$ column of $\hat{\bb{Q}}_{HQR} = \bb{Q} + \Delta \bb{Q}_{HQR}$ is bounded normwise via $\|\Delta \bb{Q}_{HQR}[:,j]\|_2 \leq n\tilde{\gamma}_{m}^{h},$ and incurs an extra rounding error when $\hat{\bb{Q}}_{HQR}\in\F_{h}^{m\times n}$ is cast down to $\F_{l}^{m\times n}$.
%Casting down $x\in\F_h$ to $\F_l$ without overflow or underflow results in \[\text{\tt castdown}(x) = x(1+\dd^{(l)}),\;\; |\dd^{(l)}| < u^{(l)},\]
%and accrues a single rounding error in the lower precision.
%Extending this result, we represent the backward error of a casting down a vector in $\F_h^{(m)}$ with a linear transformation, $\bb{I}^{(l)}\in\R^{m\times m}$.
%This transformation is a diagonal perturbation of the identity, $\bb{I}_m$.
%For some vector $\bb{x}\in\F_h$, the cast down operation yields
%\begin{equation}
%	\bb{x}^{(l)} := \text{\tt castdown}(\bb{x}^{(h)}) = \bb{I}_{l}\bb{x}^{(h)} = (\bb{I}+\bb{E})\bb{x}^{(h)} = \bb{x}^{(h)}+\Delta \bb{x},
%\end{equation}
%where $|\Delta \bb{x}| \leq u^{(l)} |\bb{x}^{(h)}|$ and  $\|\Delta \bb{x}\|_2 \leq u^{(l)} \|\bb{x}^{(h)}\|_2$.
%Then, $\bb{E} = \Delta \bb{x x}^{\top}/\|\bb{x}\|_2^2$ and we can use the same argument as in \cref{eqn:outer} to form a backward matrix norm bound, 
%\begin{equation}
%	\|\bb{E}\|_F\leq u^{(l)}. \label{eqn:castdown}
%\end{equation}
Using this in \cref{lem:3.7} to analyze the forward norm error for the $j^{th}$ column of the $\bb{Q}$ factor computed with \cref{algo:hhQR} yields
\begin{equation}
	\|(\text{\tt castdown}(\hat{\bb{Q}}_{HQR})- \bb{Q})[:,j]\|_2 = \|(\bb{I}^{(l)}\hat{\bb{P}}_{1}\cdots\hat{\bb{P}}_{n}-\bb{P}_{1}\cdots\bb{P}_{n})\hat{\bb{e}}_j\|_2 \leq u^{(l)}+n\tilde{\gamma}_m^{(h)} + nu^{(l)}\tilde{\gamma}_m^{(h)}.\label{eqn:HQR-mp}
\end{equation}
The final castdown operation increases the upper bound by $u^{(l)}$ and the size of $\bb{A}$ has no impact on this extra rounding error. 
%To convert this bound to the lower precision, we define function $d$,
%\begin{equation}
%d(m,u^{(h)}, q,u^{(l)}) := \lceil (qu^{(l)}+mu^{(h)})/u^{(l)}\rceil = \cO(q+mu^{(h)}/u^{(l)}),\label{eqn:d}
%\end{equation} 
%so that if $\|\hat{\bb{x}}-\bb{x}\|_2 \leq \gamma_m^{(h)}$, then $\|\text{\tt castdown}(\hat{\bb{x}})-\bb{x}\|_2 \leq \gamma_{d(m,u^{(h)}, q,u^{(l)})}^{(l)}$.
%This is a looser bound but it allows us to easily compare the errors to the uniform, low precision implementation of forming $\hat{\bb{x}}$.
%A looser upper bound is given by $\tilde{\gamma}_d^{(l)}$ where $d = \lceil u^{(l)}_+mu^{(h)}/u^{(l)}\rceil$ and we benefit from being able to represent it in terms of the low precision . 
%Additionally, we can use this to identify a function that allows us to formulate the new error bound after applying a castdown operation to vectors whose error bounds were known in high precision,
%Let $d = \lceil nmu^{(h)}/u^{(l)}\rceil$, so that $nmu^{(h)}\leq du^{(l)}$, and there exists some $r\leq \lfloor u^{(h)}/u^{(l)} \rfloor$ so that $nmu^{(h)} = (d-1)u^{(l)}+ru^{(h)}$.
%This $d$ value allows us to convert the error in terms of the higher precision to the lower precision while also adding in the extra rounding error that may be incurred in the casting down step. 
%Then, the componentwise error is bounded by
%\begin{equation}
%	|\text{{\tt castdown}}(\Delta \bb{Q}_{HQR}[i,j])| \leq \frac{cnmu^{(h)}}{1-cmu^{(h)}} \leq \frac{cdu^{(l)}}{1-cmu^{(h)}} \leq \tilde{\gamma}_{d}^{(l)},
%\end{equation}
%and the columnwise error is 
%\begin{equation}
%	\|\text{{\tt castdown}}(\Delta \bb{Q}_{HQR}[:,j])\|_2 \leq \left(\sum_{i=1}^m |\tilde{\tth}_d^{(l)}|^2 \right)^{1/2} \leq \sqrt{m}\gamma_d^{(l)}
%\end{equation}
%Since $\hat{\bb{Q}}_{HQR}$ should be almost orthogonal with respect to the higher precision, we can expect all components to be within the dynamic range of $\F_{l}$.
%In \ref{sec:mp-f}, we look at the rounding errors incurred from carrying out a QR factorization in a high precision, then cast down at the very end.
%Since this requires only one cast down operation, this is very similar to the results from the standard uniform precision analysis.
%Similarly, we can apply the operator $\bb{I}^{(l)}$ to cast down any quantity stored in the higher precision. 
%We consider this trivial case for HQR since BQR and TSQR use it as a subroutine. 
%If we consider the same trivial mixed precision setting for BQR and TSQR, then the corresponding forward matrix norm errors on the $\bb{Q}$ factor are
%\begin{align*}
%	\|\hat{\bb{Q}}_{BQR}\|_F&\leq u^{(l)}+n\tilde{\gamma}_m^{(h)} +u^{(l)}n\tilde{\gamma}_m^{(h)},\\
%	%\leq \tilde{\gamma}_{d(nm,u^{(h)},u^{(l)})}^{(l)},\\
%	\|\hat{\bb{Q}}_{TSQR}\|_F&\leq u^{(l)}+n(L\tilde{\gamma}_{2n}^{(h)}+\tilde{\gamma}_{m2^{-L}}^{(h)}) +u^{(l)}n(L\tilde{\gamma}_{2n}^{(h)}+\tilde{\gamma}_{m2^{-L}}^{(h)}).
%	%\leq \tilde{\gamma}_{d(n(L2n+m2^{-L}),u^{(h)},1,u^{(l)})}^{(l)}.
%\end{align*}
Applying this trivial mixed precision setting to BQR and TSQR would simply increases the error bound by approximately $u^{(l)}$ all the while taking an even longer time than the high precision implementation due the extra cast down and cast up operations.
Therefore, we do not analyze the rounding error analysis of this mixed precision variant of BQR and TSQR.
However, we will use this mixed precision HQR as a subroutine of the mixed precision BQR and TSQR in the following section. 
%We will modify BQR and TSQR so that matrix-matrix multiply and accumulate operations can be performed on TensorCore block FMAs which work on $4$-by-$4$ matrices, $\bb{A},\bb{B},\bb{C},$, and $\bb{D}$ that compute \[
%\bb{D} = \fl( \bb{C} +\bb{A}\bb{B}),\]
%where $\bb{A},\bb{B}\in \F_{\text{fp16}}^{4\times 4}$ and  $\bb{C},\bb{D}\in \F_{\text{fp16}}^{4\times 4}$ or $\bb{C},\bb{D}\in \F_{\text{fp32}}^{4\times 4}$.
%The inner product step in forming $\bb{A}\bb{B}$ is similar to \cref{assump:mp} in that full precision (exact) products are accumulated in the higher precision, fp32.
%One difference is that the cast down operation at the end of the inner product is optional.
%Matrices larger than $4$-by-$4$'s can be multiplied and added using this optional cast down feature and by using block matrix multiplication with $4$-by-$4$ blocks.
%In \cref{sec:mp-3}, we consider performing BQR and TSQR with high precision FLOPs within a block/level, but cast down to low precision in between blocks and at the very end.
%Finally, in \cref{sec:mp-2}, we consider all 3 algorithms with the ad hoc mixed precision setting described in \cref{assump:mp} where inner products are performed in high precision before being cast down, and all other operations are computed in low precision.
%\subsection{Round down at the end of the factorization}\label{sec:mp-f}
%Results in this section are quite straight forward, but 
%\subsubsection{HQR}
\subsection{Round down at block-level (BLAS-3)}\label{sec:mp-3}
The mixed precision setting in this section is designed to meet the below requirements.
\begin{enumerate}
	\item Modify \Cref{algo:blockHQR,algo:par_tsqr} to maximize level-3 BLAS operations and use TensorCore bFMAs. 
	\item Apply \cref{eqn:HQR-mp} to all instances of HQR to the error analyses for BQR and TSQR in \cref{sec:algo}.
	\item Cast down quantities at every block/level and the insertion of low precision errors $u^{(l)}$ should be somewhat correlated to the number of blocks and levels. 
	\item Both input and output of the various QR factorization algorithms are given in the low precision. 
\end{enumerate}
TensorCore's bFMA computes 
\begin{equation}
\hat{\bb{D}} =\fl_{TC}(\bb{C} + \bb{A}\bb{B}),\qquad \bb{C},\bb{D}\in\F_{\text{fp16}}^{4\times 4}\text{ or }\F_{\text{fp32}}^{4\times 4},\text{ and } \bb{A},\bb{B}\in\F_{\text{fp16}}^{4\times 4},\label{eqn:bFMA}
\end{equation}
and employs \emph{full} precision products and fp32 summation accumulate.
Here, the \emph{full} precision multiplication is exact as explained in \cref{sec:background}.
In \cite{Blanchard2019}, the authors investigate all four possible matrix-matrix multiplication routines in TensorCore, which depend on whether $\bb{C}$ and $\bb{D}$ are computed in fp16 or fp32. 
They also note that matrices larger than $4$-by-$4$ can still be computed using this block FMA by accumulating matrix sums with $\bb{C}\in\F_{\text{fp32}}^{4\times 4}$.
Suppose that we aim to compute a fp16 matrix product of two fp16 matrices, $\bb{X}\in\F_{(fp16)}^{m\times p}$, $\bb{Y}\in\F_{(fp16)}^{p\times n}$, and $\bb{Z}=\bb{XY}\in\F_{\text{fp16}}^{m\times n}$.
We pad $\bb{X},\bb{Y}$ with zeros so that all matrix dimensions are multiples of $4$ and the matrix product can be computed with the TensorCore block FMA.
Let $\bb{Q}_{[i,j]}:= \bb{Q}[4(i-1)+1:4i,4(j-1)+1:4j]$ refer to the $(i,j)^{th}$ $4$-by-$4$ block for any $\bb{Q}\in\{\bb{X},\bb{Y},\bb{Z}\}$.
Then, we compute $\bb{Z}_{[i,j]}$ via \[
\bb{Z}_{[i,j]} = \sum_{k=1}^{\lceil p/4\rceil} \bb{X}_{[i,k]} \bb{Y}_{[k,j]},
\]
where we use \cref{eqn:bFMA} by initializing with $\bb{A}^{(1)}:= \bb{X}_{[i,1]}$, $\bb{B}^{(1)}:= \bb{Y}_{[1,j]}$, and $\bb{C}^{(1)}:= \bb{0}_{4\times 4}$ and setting $\bb{A}^{(k)}:= \bb{X}_{[i,k]}$, $\bb{B}^{(k)}:= \bb{Y}_{[k,j]}$, and $\bb{C}^{(k)}:= \bb{D}^{(k-1)}$ for $k=2:\lceil p/4\rceil$.
By setting $\bb{C}^{(k)}, \bb{D}^{(k)}\in\F_{\text{fp32}}^{4\times 4}$ for $k>1$ and only casting down at the end via $\bb{Z}_{[i,j]} =$ fp16$(\bb{D}^{(\lceil p/4\rceil)})$, we mostly employ fp32 arithmetic for a mixed precision matrix product routine whose inputs and output are in fp16.
For example, take $p=8$.
Then,
\begin{align*}
\bb{D}^{(1)} &= \fl_{TC}(\bb{X}_{[i,1]} \bb{Y}_{[1,j]}),\quad\bb{D}^{(2)} = \fl_{TC}(\bb{X}_{[i,2]} \bb{Y}_{[2,j]} + \bb{D}^{(1)})\in\F_{\text{fp32}}^{4\times 4}\\
\bb{Z}_{[i,j]} &= \text{\tt castdown}(\bb{D}^{(2)})\in\F_{\text{fp16}}^{4\times 4}.
\end{align*}
Adapting the rounding error analysis in \cite{Blanchard2019} into this specific mixed precision matrix product setting yields the componentwise forward bound 
\begin{equation}
|\bb{Z}-\fl(\bb{Z})| \leq \left(u^{(\text{fp16})}+ \gamma_{p/4}^{(\text{fp32})}+u^{(\text{fp16})} \gamma_{p/4}^{(\text{fp32})}\right)|\bb{X}||\bb{Y}|.\label{eqn:bFMAerr}
\end{equation}

We denote BQR and TSQR computed via TensorCore bFMA's with {\tt mpBQR3} and {\tt mpTSQR3}, where the {\tt 3} represents the BLAS level-3 nature of this mixed precision setting.
\subsubsection{Round down at block level: BQR}\label{sec:mp-3b}
%Let us consider a setting in which only $M$ blocks of width $r$ can be loaded onto memory.
%Then, lines 2-6 of \cref{algo:blockHQR} can be modified via \cref{algo:mpBQR}.
%\begin{algorithm2e}
%	$q = N/M$\tcc*{Note that $n=Nr=qMr$.}
%	\For{$q'=1:q$}{
%		\If {$q'>2$} {
%				Update $[\bb{C}_{(q'-1)M+1}\cdots\bb{C}_{qM}]$ with WY updates from blocks $1:(q'-1)M$.
%		}
%		\For{$k=1:M$}{
%		Apply HQR to $\bb{C}_{(q'-1)M+k}$\;
%		Form WY update for $\bb{C}_{(q'-1)M+k}$\;
%		WY update blocks to the right, $[\bb{C}_{(q'-1)M+k+1}\cdots \bb{C}_{q'M}]$.
%	}
%}
%\caption{\label{algo:mpBQR} A portion of a mixed precision BQR: modifying first for-loop in \cref{algo:blockHQR}.}
%\end{algorithm2e}
%We now impose a mixed-precision setting where the inner for-loop in \cref{algo:mpBQR} is performed in high precision, but the WY updates for the outer loop is stored in low precision and only $M$ blocks is updated at a time due to the memory constraint.
%These low precision WY updates would be used to build the $\bb{Q}$ factor serially in groups of $M$.
Consider the input matrix, $\bb{A}\in\F_l^{m\times n}$, partitioned into $N$ blocks of $r$ columns, $\bb{A}=[\bb{C}_1 \cdots \bb{C}_N]$ as in \cref{sec:BQR}.
%Since \cref{algo:blockHQR} uses level-3 BLAS operations on $1-\cO(1/N)$ fraction of all the FLOPs, we modify it so that those operations are performed with TensorCore bFMAs. 
%%We assume that the returned factors should also be represented in the lower precision, $\F_l$, and modify \cref{algo:blockHQR} so that matrix-matrix multiply and accumulate operations are performed with TensorCore block FMAs.
%The remaining FLOPs are a relatively small($\cO(1/N)$) fraction in comparison, and we assume that we can afford to compute these level-1 and 2 BLAS operations using high precision.
\Cref{algo:mpBQR} shows a mixed precision variant of BQR that maximizes the use of bFMAs but uses high precision arithmetic for level-1 and 2 BLAS operations which are only a $\cO(1/N)$ fraction of the total number of FLOPs. 
Each block is casted up to compute a high precision HQR and to form the WY representation. 
The WY representation is then casted down to low precision since the bFMAs require low precision inputs for matrix products, and the $\bb{R}$ factor from the high precision HQR can be casted down to return a low precision $\bb{R}$ factor at the very end. 
Since the cast down operations for the $\bb{R}$ factor and the WY representations occur at every block, we can expect columnwise error bound for 
\cref{algo:mpBQR} to increase by approximately $Nu^{(l)}$ from the error bound for \cref{algo:blockHQR}.
%The $\bb{R}$ factor from each call to HQR is stored in low precision, but the HH constants and vectors ($\bm{\beta}_k^{(j)}$,$\bb{v}_k^{(j)}$) are kept in high precision to build the WY representation.
%We enforce a cast down at the end of \cref{algo:buildWY} since the bFMAs require low precision inputs.
%Since the WY representations ($\bb{W}_k$, $\bb{V}_k$) should be stored in low precision, we enforce a cast down at the end of \cref{algo:buildWY}.
%Finally, all but the last WY update for each block are stored in the higher precision, and the last WY update returned in low precision. 
%This mixed precision BQR variant is rich in level-3 BLAS operations can be implemented with TensorCore block FMAs easily, and is formally introduced in \cref{algo:mpBQR}.
\begin{algorithm2e}
	\DontPrintSemicolon % Some LaTeX compilers require you to use \dontprintsemicolon instead
	\KwIn{$\bb{A}$, $r$. \hfill\textbf{Output: }$\hat{\bb{Q}}_{mpBQR3}$,$\hat{\bb{R}}_{mpBQR3}$}
	$N=\frac{n}{r}$\tcc*{Let $\bb{A} = [\bb{C}_{1} \cdots  \bb{C}_{N}]$.}
	%\tcp{Let $n_i=ri$ for $i=1:N-1$ and $n_N=n$.} 
	\For{$k=1:N-1$}{
%		\If{$k == 1$}{
			$\bb{V}_{k},\bm{\beta}_k,\bb{C}_{k}\gets$ {\tt hhQR}({\tt castup}($\bb{C}_{k}$))\tcc*{\Cref{algo:hhQR} in high precision.}
%		}
%		\Else{
%		$\bb{V}_{k},\bm{\beta}_k,\bb{C}_{k}\gets$ {\tt hhQR}($\bb{C}_{k}$)\tcc*{\Cref{algo:hhQR} in high precision.}	
%	}
		$\bb{C}_{k}\gets ${\tt castdown }($\bb{C}_{k}$)\tcc*{Builds $\bb{R}$ factor in low precision.}
		%$\bb{V}_i,\bm{\beta}_i,\bb{A}_{n_{i-1}+1:m,n_{i-1}+1:n_i}\gets$ {\tt hhQR}$(\bb{A}_{n_{i-1}:m,n_{i-1}+1:n_i})$\tcc*{\Cref{algo:hhQR}}
		$\bb{W}_{k}\gets $ {\tt buildWY}$(\bb{V}_{k},\bm{\beta}_k)$ \tcc*{\Cref{algo:buildWY} in high precision}
		$[\bb{V}_{k},\bb{W}_{k}]\gets ${\tt castdown}($[\bb{V}_{k},\bb{W}_{k}]$)\;
		$[\bb{C}_{k+1}\cdots\bb{C}_{N}]$ -= $\bb{V}_{k} \left(\bb{W}_{k}^{\top}[\bb{C}_{k+1}\cdots\bb{C}_{N}]\right) $ \tcc*{returned in low precision}
	}
	%\tcp{Now build $\bb{Q}$ using level-3 BLAS operations.} 
	$\bb{Q}\gets \bb{I}$\tcc*{Build $\bb{Q}$: $\bb{I}_m$ if full QR, and $\bb{I}_{m\times n}$ if thin QR.}
	\For{$k=N:-1:1$}{
		\tcp{All updates are returned in low precision.}
		$\bb{Q}[(k-1)r+1:m,(k-1)r+1:n]$-= $\bb{W}_k \left(\bb{V}_k^{\top}\bb{Q}[(k-1)r+1:m,(k-1)r+1:n]\right)$
	}
	%\tcp{The last update is returned in low precision.}
	%$\bb{Q}$-= $\bb{W}_1 \left(\bb{V}_1^{\top}\bb{Q}\right)$\;
	\Return $\bb{Q},\bb{A}$
	\caption{\label{algo:mpBQR} $\hat{\bb{Q}}_{mpBQR3},\hat{\bb{R}}_{mpBQR3}\gets {\tt mpBQR3}(\bb{A}, r)$: Perform a mixed precision variant of BQR of low precision $\bb{A}$ with column partitions of size $r$. $\hat{\bb{Q}}_{mpBQR3}$,$\hat{\bb{R}}_{mpBQR3}$, and $\hat{\bb{A}}$ are represented in low precision. Matrix-matrix multiplication and accumulate operations in lines 10, 13, and 14 require low precision inputs but can return in either of the two precisions.}
\end{algorithm2e}

\paragraph{Rounding Error Analysis} Since $\hat{\bb{W}}_{k},\hat{\bb{Y}}_k$'s are computed with \cref{algo:buildWY} in high precision then cast down, the new low precision WY update is $\hat{\bb{X}}_{k}^{(l)} = \bb{I}-\bb{I}^{(l)}\hat{\bb{W}}_k\bb{I}^{(l)}\hat{\bb{V}}_k^{(\top)}$.
Consider applying $\hat{\bb{X}}_k^{(l)}$ to some matrix stored in low precision, $\bb{B}$ using the TensorCore bFMAs.
We analyze a single column $\bb{b}_j:=\bb{B}[:,j] \in \F_l^{m-(k-1)r}$ even though this operation is done on $\bb{B}$ as a whole.
Let \[\bb{I}^{(l)}\hat{\bb{W}}_k = (\bb{I}+\bb{E}_W)\hat{\bb{W}}_k,\quad \bb{I}^{(l)}\hat{\bb{Y}}_k = (\bb{I}+\bb{E}_Y)\hat{\bb{Y}}_k,\] where $\bb{E}_W,\bb{E}_Y$ are diagonal and bounded componentwise by $u^{(l)}$.  
%Since \[\hat{\bb{X}}_{k}^{(l)}- \bb{X}_k = \hat{\bb{X}}_{k}^{(l)} -\hat{\bb{X}}_{k}+ \hat{\bb{X}}_{k}- \bb{X}_k= \hat{\bb{X}}_{k}^{(l)} -\hat{\bb{X}}_{k}+ \Delta \bb{X}_{k},\]
%the errors derived in \cref{eqn:deltX} since the bFMAs perform block $4$-by-$4$ matrix products using \cref{eqn:bFMAerr}, then.
The rounding errors for forming $\bb{W}_k$ and $\bb{Y}_k$ remain the same since these are computed in high precision. 
Therefore, we first include errors introduced from casting down the WY representation and compute the matrix norm error of forming $\hat{\bb{X}}_{k}^{(l)}$,
%to $\bb{b}_j$ with exact arithmetic.
\begin{align*}
	\|\hat{\bb{X}}_{k}^{(l)}- \bb{X}_{k}\|_F  &= \|-\left(\bb{I}+\bb{E}_W+\bb{E}_Y+ \bb{E}_W\bb{E}_Y\right)\hat{\bb{W}}_k\hat{\bb{Y}}_k^{\top} + \bb{W}_k\bb{Y}_k^{\top}\|_F,\\
	&\leq \left((1+\gamma_2^{(l)}+(u^{(l)})^2)r\tilde{\gamma}_{m-(k-1)r}^{(h)}+\gamma_2^{(l)}+(u^{(l)})^2\right)\|\bb{X}_k\|_F\\
	&\leq \tilde{\gamma}_2^{(l)} +r\tilde{\gamma}_{m-(k-1)r}^{(h)} + r\tilde{\gamma}_2^{(l)}\tilde{\gamma}_{m-(k-1)r}^{(h)}.
\end{align*}
Now, we consider the backward error of applying $\hat{\bb{X}}_{k}^{(l)}$ to $\bb{b}_j$ with the bFMA matrix product error bound from \cref{eqn:bFMAerr}.
The multiplication by $(\bb{I}^{(l)}\hat{\bb{Y}}_k)^{\top}$ yields backward error bounded by
\begin{equation*}
	\fl_{TC}((\bb{I}^{(l)}\hat{\bb{Y}}_k)^{\top}\bb{b}_j) = (\hat{\bb{Y}}_k+\Delta_{TC}\hat{\bb{Y}}_k)\bb{b}_j,\;\;|\Delta_{TC}\hat{\bb{Y}}_k| \leq u^{(l)}+\gamma_{\frac{m-(k-1)}{4}}^{(h)}+u^{(l)}\gamma_{\frac{m-(k-1)}{4}}^{(h)}|\hat{\bb{Y}}_k||\bb{b}_j|,
%	\fl_{TC}(\bb{b}_j-\bb{W}_k\fl_{TC}((\bb{I}^{(l)}\hat{\bb{Y}}_k)^{\top}\bb{b}_j)) &= (\hat{\bb{X}}_k+\Delta_{TC}\hat{\bb{X}}_k)\bb{b}_j,
\end{equation*}
and the subsequent multiplication by $(\bb{I}^{(l)}\hat{\bb{W}}_k)$ and subtraction from $\bb{b}_j$ result in,
% $\cO(u^{(l)}+\gamma_{1+r/4}^{(h)}+u^{(l)}\gamma_{1+r/4}^{(h)})$ to the error bound.
%Finally, this is summarized by
\begin{align*}
	\fl_{TC}(\hat{\bb{X}}_{k}^{(l)}\bb{b}_j) &= (\hat{\bb{X}}_{k}^{(l)}+\Delta^{(l)}\bb{X}_k)\bb{b}_j,\\
	|\Delta^{(l)}\bb{X}_k| &\leq \left(\gamma_2^{(l)}+\gamma_{1+\frac{m-(k-2)r}{4}}^{(h)}+\gamma_2^{(l)}\gamma_{1+\frac{m-(k-2)r}{4}}^{(h)}\right)\left(|\bb{b}_j|+|\bb{I}^{(l)}\hat{\bb{W}}_k||\bb{I}^{(l)}\hat{\bb{Y}}_k|^{\top}|\bb{b}_j|\right).
\end{align*}
Converting to a normwise error bound using the same logic from \cref{eqn:applyP,eqn:19.2c}, we result in  
%Recall that applying $\bb{X}_k$ in high precision  
%\begin{align*}
%	\|(\hat{\bb{X}}_{k}^{(l)}- \hat{\bb{X}}_{k}+ \Delta \bb{X}_{k})\bb{b}_j \|_2 %&=\| (\bb{b}_j-(\hat{\bb{W}}_k + \bb{E}_W\hat{\bb{W}}_k)((\hat{\bb{Y}}_k + \bb{E}_Y\hat{\bb{Y}}_k)^{\top}\bb{b}_j)) - (\bb{b}_j-\hat{\bb{W}}_k(\hat{\bb{Y}}_k^{(\top)}\bb{b}_j) + \Delta \bb{X}_{k} \bb{b}_j\|_2\\
%	&= \| \left(-\left(\bb{E}_W+\bb{E}_Y+ \bb{E}_W\bb{E}_Y\right)\hat{\bb{W}}_k\hat{\bb{Y}}_k^{\top} + \Delta \bb{X}_{k} \right)\bb{b}_j\|_2,\\
%	&= (\gamma_2^{(l)}(1+r\tilde{\gamma}_{m-(k-1)r}^{(h)}) + r\tilde{\gamma}_{m-(k-1)r}^{(h)}) \|\bb{b}_j\|_2.
%%	\text{\tt castdown}(\hat{\bb{P}}_{k'}^{(1)}\cdots\hat{\bb{P}}_{k'}^{(r)}) \\
%%	\|\hat{\bb{X}}_{k'} - \bb{X}_{k'}\|_F &= \|\bb{I}^{(l)}\hat{\bb{P}}_{k'}^{(1)}\cdots\hat{\bb{P}}_{k'}^{(r)} -\bb{P}_{k'}^{(1)}\cdots\bb{P}_{k'}^{(r)} \|_F \leq (1+u^{(l)})(1+r\tilde{\gamma}_{m-(k'-1)r})-1,
%\end{align*}
\begin{equation}
\|\fl_{TC}(\hat{\bb{X}}_{k}^{(l)}\bb{b}_j)-\bb{X}_k\bb{b}_j\|_2 \leq (\tilde{\gamma}_2^{(l)} +r\tilde{\gamma}_{m-(k-1)r}^{(h)} + r\gamma_2^{(l)}\tilde{\gamma}_{m-(k-1)r}^{(h)}) \|\bb{b}_j\|_2, 
\end{equation}
since the rounding errors from the bFMAs are small in comparison to the errors from casting down the WY representation built in high precision.
The corresponding matrix error bound is
\begin{equation}
	\|\fl_{TC}(\hat{\bb{X}}_{k}^{(l)})-\bb{X}_k\|_F \leq \tilde{\gamma}_2^{(l)} +r\tilde{\gamma}_{m-(k-1)r}^{(h)} + r\tilde{\gamma}_2^{(l)}\tilde{\gamma}_{m-(k-1)r}^{(h)}.\label{eqn:mpdeltX}
\end{equation}
%where $\Delta^{(l)}\bb{X}_k = \fl_{TC}(\hat{\bb{X}}_{k}^{(l)})-\bb{X}_k$.

We can finally compute the forward errors from implementing \cref{algo:mpBQR}.
Consider the $j^{th}$ column of the $\bb{Q}$ factor, which we denote with $\bb{q}_j:=\hat{\bb{Q}}_{mpBQR3}[:,j]$, and let $k = \lfloor j/r\rfloor$.
Invoking \cref{lem:3.7} with error bounds for $\fl_{TC}(\hat{\bb{X}}_k^{(l)})$'s in \cref{eqn:mpdeltX} results in columnwise error,
\begin{align}
	\|\Delta \bb{q}_j \|_2 &\leq -1 + \prod_{k'=1}^k (1+\tilde{\gamma}_2^{(l)})(1+r\tilde{\gamma}_{m-(k'-1)r}^{(h)})\\ 
	&\leq k\tilde{\gamma}_{2}^{(l)} + kr\tilde{\gamma}_m^{(h)} + k^2r\tilde{\gamma}_{2}^{(l)}\tilde{\gamma}_m^{(h)}, \label{eqn:mpBQRcol}
\end{align} 
where $\Delta \bb{q}_j = (\fl_{TC}(\hat{\bb{X}}_1^{(l)})\cdots\fl_{TC}(\hat{\bb{X}}_k^{(l)}) - \bb{X}_1\cdots\bb{X}_k )\hat{\bb{e}}_j.$
Summing over the columns to find a matrix norm error bound yields
\begin{equation}
	\|\hat{\bb{Q}}_{mpBQR}-\bb{Q}\|_F \leq n^{1/2}\tilde{\gamma}_{N}^{(l)} + n^{(3/2)}\tilde{\gamma}_m^{(h)},
\end{equation}
where the summation of the third term in \cref{eqn:mpBQRcol} is swept under the tilde notation in $n^{1/2} \tilde{\gamma}_{N}^{(l)}$.
This bound shows that \cref{algo:mpBQR} only adds $n^{1/2}\tilde{\gamma}_{N}^{(l)}$ order errors to the bounds in \cref{eqn:BQRmat}.
Using that $u^{(l)}=M_{l,h}u^{(h)}$, this increase corresponds to a multiplicative factor shown below,
\begin{equation}
	n^{1/2}\tilde{\gamma}_{N}^{(l)} + n^{(3/2)}\tilde{\gamma}_m^{(h)} \approx \left(1+\frac{M_{l,h}}{rm}\right)n^{(3/2)}\tilde{\gamma}_m^{(h)}. \label{eqn:mpBQR3}
\end{equation}
Therefore, the loss in accuracy due to mixed precision computing is relatively small when the disparity in precision ($M_{l,h}$) is small in comparison to the block size, $mr$.
Whether this loss in accuracy in the worst-case scenario is worth the speed-ups from using mixed precision hardware is an open question that can be tackled in future research.
We expect that the block size $r$, the dimension of the input matrix $m,n$, and hardware specificities will be contributing factors. 

\subsubsection{Round down at block level: TSQR}\label{sec:mp-3t}
%Unlike BQR, which is rich in level-3 BLAS operations, TSQR d
%Since the majority of FLOPs in BQR use level-3 BLAS operations, it was simple to adapt TensorCore block FMAs within the algorithm.
%
%TSQR is best suited for tall-and-skinny matrices where $m\gg n$. 
%The majority of the computational gains are from parallelizing the QR factorizations of the initial level blocks, which are performed by different node/tasks ideally.
Unlike BQR which is rich in level-3 BLAS operations, the variant of TSQR in \cref{algo:par_tsqr} uses none.
Therefore, we modify \cref{algo:par_tsqr} by replacing all instances of {\tt hh\_mult} with level-3 BLAS operations.
We omit presenting the exact algorithm for mixed precision variant of TSQR in this paper, but consider computing the HQR of each block in high precision and build and store the WY representation of the HH transformations in low precision as we did in lines (3-6) of \cref{algo:mpBQR}.
The low precision WY representation is then applied with TensorCore bFMAs when building the $\bb{Q}$ factor (lines 11-16 of \cref{algo:par_tsqr}). 

%We consider applying $n$ HH transformations with level-3 BLAS operations by building the WY representation using high precision arithmetic, casting them down and then applying the update with TensorCore block FMAs as in lines 3-6 of \cref{algo:mpBQR}.
%but for our mixed precision bFMA TSQR variant, the WY transformations are only used when building the $\bb{Q}$ factor.
%Note that for all blocks in all levels, exactly $n$ HH transformations of lengths either $m2^{-L}$ or $2n$ are applied via {\tt hh\_mult}, and let $m' := \max\{m2^{-L},2n\}$ be the larger of the two. 
%The multiplication by the $\bb{Y}$ factor requires at most $m'$-length inner products and a cast down operation and the multiplication by the $\bb{W}$ factor requires $n$-length inner products and another cast down operation.
%The errors accumulated from these actions are bounded by $\cO(3u^{(l)}+\tilde{\gamma}_{m'n}^{(h)})$ componentwise.
%	%\item Now, consider applying the $n$ HH transformations by forming the operator explicitly then computing a single matrix-matrix product.
%	We form the operator using the same steps as forming the $\bb{Q}$ factor in HQR in high precision arithmetic, then cast the result down. 
%	The construction of the operator and the cast down result in error bounded by $\cO(u^{(l)}+\tilde{\gamma}_{m'n}^{(h)})$, and the matrix-matrix product requires $m'$-length inner products and another cast down operation. 
%	This option also accrues error bounded by $\cO(2u^{(l)}+\tilde{\gamma}_{m'n}^{(h)})$.
%\end{enumerate}
%This requires more FLOPs than in the standard algorithm implemented with level-2 BLAS operations, since the same number of level-2 BLAS operations are required to form the matrices required for the level-3 variants. 
%The level-3 variants are built during the formation of the $\bb{R}$ factor and can be reused when forming the $\bb{Q}$ factor.
%Notice that unlike BQR which is rich in level-3 BLAS operations by design, it is not clear whether implementing TSQR with level-3 BLAS operations has obvious benefits.
%Regardless, we still perform a rounding error analysis of TSQR performed with mixed precision level-3 BLAS operations, i.e. TensorCore block FMAs.
\paragraph{Rounding Error analysis} The analysis in \cite{Mori2012} shows that each column of $\bb{Q}$ is transformed by $n$ HH transformations of length $2n$ from levels $L:-1:1$, and another set of $n$ HH transformations of length $m2^{-L}$ at level $0$.
Let us represent the WY representation at the $j^{th}$ block of level $i$ and its bFMA counterpart as $\bb{X}_j^{(i)}$ and $\fl_{TC}(\hat{\bb{X}}_j^{(i)})$.
Then, we can use \cref{eqn:mpdeltX} to form backward error  
\begin{equation}
	\|\fl_{TC}(\hat{\bb{X}}_j^{(i)})-\bb{X}_j^{(i)})\|_F \leq \tilde{\gamma}_2^{(l)} +n\tilde{\gamma}_{m'}^{(h)} + n\tilde{\gamma}_2^{(l)}\tilde{\gamma}_{m'}^{(h)}, \;\; m' = \begin{cases}
	m2^{-L}, &i=0\\
	2n, & i = 1 : L
	\end{cases}.
\end{equation}
% and apply \cref{lem:3.7}$(1+2u^{(l)})$ to every set of $n$ HH transformations in each level.
%There are two low precision rounding errors in each block per level since casting down the matrix operator formed with high precision is cast down to the low precision, and the matrix-matrix product used in applying this operator with TensorCore block FMAs incurs another low precision rounding error. 
We can now modify the analysis in \cite{Mori2012} by replacing $n\tilde{\gamma}_{m2^{-L}}$ and $n\tilde{\gamma}_{2n}$ with \[(1+\tilde{\gamma}_2^{(l)})(1+n\tilde{\gamma}_{m2^{-L}}^{(h)})-1,\quad\text{and}\quad (1+\tilde{\gamma}_2^{(l)})(1+n\tilde{\gamma}_{2n}^{(h)})-1,\]
and apply \cref{lem:3.7}.
Then, the $\bb{Q}$ factor formed with this mixed precision variant of TSQR is denoted with $\hat{\bb{Q}}_{mpTSQR3}$ and its $j^{th}$ column has rounding errors bounded by,
\begin{equation}
\|\hat{\bb{Q}}_{mpTSQR3}[:,j] - \bb{Q}[:,j]\|_2 \leq \tilde{\gamma}_{L+1}^{(l)}+n\left(L\tilde{\gamma}_{2n}^{(h)}+\tilde{\gamma}_{m2^{-L}}^{(h)}\right)\label{eqn:mpTSQR1}.
\end{equation}
Summing up the columns for a matrix norm error bound, we result in 
\begin{equation}
	\|\hat{\bb{Q}}_{mpTSQR3} - \bb{Q}\|_F \leq n^{1/2}\tilde{\gamma}_{L+1}^{(l)}+n^{3/2}\left(L\tilde{\gamma}_{2n}^{(h)}+\tilde{\gamma}_{m2^{-L}}^{(h)}\right).
\end{equation}
Once again, we convert the low precision rounding errors  as a fraction of the original bound in \cref{eqn:tsqrQ} to quantify the impact of modifying \cref{algo:par_tsqr} to utilize bFMAs,
\begin{equation}
	n^{1/2}\tilde{\gamma}_{L+1}^{(l)}+n^{3/2}\left(L\tilde{\gamma}_{2n}^{(h)}+\tilde{\gamma}_{m2^{-L}}^{(h)}\right) = \left(1+ \frac{M_{l,h}L}{n(2nL+m2^{-L})}\right)n^{3/2}\left(L\tilde{\gamma}_{2n}^{(h)}+\tilde{\gamma}_{m2^{-L}}^{(h)}\right).\label{eqn:mpTSQR3}
\end{equation}
Like in \cref{eqn:mpBQR3}, the constant that represents the disparity in the two precisions, $M_{l,h}$ is compared against the original matrix size $m,n$ and the block size specifications derived from $L$.
%TODO: conclusion maybe a parameter regime analysis for m, n, L, M?
%Let us now consider a WY variant of TSQR, where all instances of {\tt qr} (lines 6,7,9 of \cref{algo:par_tsqr}) are followed by {\tt buildWY} (see \cref{algo:buildWY}), and all instances of {\tt hh\_mult} is replaced by a WY update (line 6 of \cref{algo:blockHQR}).
%We additionally impose a mixed precision assumption similar to \cref{sec:mp-3b}, where we store all WY representations of HQR within the for-loop (lines 4-8) of \cref{algo:par_tsqr} in low precision, and consider the construction of the $\bb{Q}$ factor.
%We can assume that each $2n$-by-$n$ and $m2^{-L}$-by-$n$ size matrices can fit into memory and only introduce one cast down for each $\bb{Q}_j^{(i)}$ block, where $i=1:L-1$ and $j=1:2^{i-1}$.
%Let us compute lines 9-10 in the higher precision, which introduces an error of order $n\tilde{\gamma}_{2n}^{(h)}$.
%In levels $L-1$ to $1$, each WY update adds error $u^{(l)}+n\tilde{\gamma}_{2n}^{(h)}$, and the final construction at the $0^{th}$ level (line 16), the WY update adds error $u^{(l)} + n\tilde{\gamma}_{m2^{-L}}^{(h)}$.
\subsection{Round down at inner product: level-2 BLAS mixed precision setting}\label{sec:mp-2}
While the previous section discussed blocked variants of HQR that can be easily adapted for the mixed precision setting specific to TensorCore bFMA's, we want to provide a more general mixed precision environment in this section.
Recall that HQR, BQR, and TSQR all rely on HH transformations in one way or another, and implementations of HH transformations are expressed by \cref{eqn:effH}.
This implementation capitalizes on the rank-1 update structure of HH transformations where the predominant share of FLOPs is spent on an inner product, and computing the HH vector and constant also rely heavily on inner products.
Therefore, nearly all of the computational tasks for \cref{algo:hhQR,algo:blockHQR,algo:par_tsqr} are attributed to the inner product, which is important in other linear algebra tools such as projections, matrix-vector, and matrix-matrix multiply.
Consequently, we return to \cref{assump:mp}, where every inner product is cast down to the lower precision as shown in \cref{eqn:aftercd}. 
We denote HQR, BQR, and TSQR computed with \cref{assump:mp} with {\tt mpHQR2}, {\tt mpBQR2}, and {\tt mpTSQR2}, where the {\tt 2} represents the mixed precision procedure computed at a level-2 BLAS operation.
%\cref{sec:background}, where every inner product is cast down to the lower precision as shown in \cref{eqn:aftercd}.
\subsubsection{HQR round down at inner product: {\tt mpHQR2}}
Consider forming a HH transformation that zeros out $\bb{x}\in\R^m$ below the the $i^{th}$ element. 
We need to compute $\sigma$, $\beta$, $\tilde{\bb{v}}_1$, and $\bb{v}$ as defined in \cref{sec:HQR},
\begin{align}
\fl(\sigma) &= \fl(-\rm{sign}(\bb{x}[1])\|\bb{x}\|_2) = \sigma + \Delta \sigma,\;\;|\Delta\sigma| \leq \left(\gamma_{2}^{(l)}+\gamma_{m}^{(h)}+\gamma_{2}^{(l)}\gamma_{m}^{(h)}\right)|\sigma|,\label{eqn:mpsigma}\\
\fl(\bb{v}'[1])& =\bb{v}'[1] + \Delta \bb{v}'[1] = (1+\dd^{(l)}) (\bb{x}[1]-\sigma-\Delta\sigma), \;\;|\Delta\bb{v}'[1]| \leq (\gamma_{3}^{(l)}+\tilde{\gamma}_{m}^{(h)})|\bb{v}'[1]| \label{eqn:mpv1}\\
\fl(\beta) &= \beta +\Delta \beta= (1+\dd^{(l)})\left(-\bb{v}'[1]/\hat{\sigma}\right), \;\; |\Delta\beta| \leq (\gamma_{8}^{(l)}+\tilde{\gamma}_{m}^{(h)})|\beta|, \label{eqn:mpbeta}\\
%\end{align}
%\begin{align}
	\fl(\bb{v}[j])	&= \bb{v}[j] + \Delta \bb{v}[j]\text{ where }|\Delta \bb{v}[j]|\leq 
%	\begin{cases}
%	0,& j=1\\
	(\gamma_{7}^{(l)} + \tilde{\gamma}_{m}^{(h)})|\bb{v}_j|,j=2:m-i+1 \label{eqn:mpv}.
%	\end{cases}  
\end{align}
These bounds on $\Delta\sigma$, $\Delta \bb{v}'[1]$, $\Delta \beta$, and $\Delta \bb{v}[j]$ are computed by using the rules from \cref{lem:mp} on the analysis shown in \cref{sec:HQR}.
Using these, we can formulate the mixed precision version of \cref{eqn:applyP} where $\hat{\bb{y}}=\fl(\bb{P_vx})\in\R^m$ is implemented via \cref{eqn:effH}.
Note that the inner product $\hat{\bb{v}}^{\top}\bb{x}$ via \cref{assump:mp}, and all other operations are done in the lower precision.
Then, the transformed vector is bounded by
\begin{equation}
	\hat{\bb{y}} = \bb{y}+\Delta \bb{y},\;\; \|\Delta \bb{y}\|_2 \leq (\gamma_{25}^{(l)} + \tilde{\gamma}_{m}^{(h)})\|\bb{y}\|_2.\label{eqn:mpdelty}
\end{equation}
Thus, a backward error can be formed using $\Delta \bb{P_v} = \Delta \bb{y}\bb{x}^{
\top}/\|\bb{x}\|_2^2$,
\begin{equation}
	\hat{\bb{y}} = (\bb{P_v} + \Delta \bb{P_v})\bb{x},\;\; \|\Delta \bb{P_v}\|_F\leq (\gamma_{25}^{(l)} + \tilde{\gamma}_{m}^{(h)}). \label{eqn:mpapplyP}
\end{equation}
Now, we form the error bounds for applying $n$ HH transformations to $\bb{x}$ using \cref{lem:3.7},
\begin{align}
\hat{\bb{z}} &= \fl(\bb{P}_1\cdots\bb{P}_n\bb{x})=\bb{Q} (\bb{x} +\Delta \bb{x}) = (\bb{Q} + \Delta \bb{Q})\bb{x},\\
\|\Delta \bb{y}\|_2 &\leq (\tilde{\gamma}_n^{(l)}+n\tilde{\gamma}_m^{(h)})\|\bb{x}\|_2,\;\; \|\Delta \bb{Q}\|_F\leq (\tilde{\gamma}_n^{(l)}+n\tilde{\gamma}_m^{(h)}).\label{eqn:mp19.3}
\end{align} 
Note that we use the $\tilde{\gamma}^{(l)}$ notation, where the small integer $c$ is now required to be $\cO(25)$.
The analogous mixed precision QR factorization error bounds are shown in \cref{thm:mpHQR}.
\begin{theorem}
	\label{thm:mpHQR}
	Let $\bb{A}\in\R^{m\times n}$ with $m\geq n$ have full rank, $n$. 
	Let $\hat{\bb{Q}}_{mpHQR2}\in\R^{m\times n}$ and $\hat{\bb{R}}\in\R^{n\times n}_{mpHQR2}$ be the thin QR factors of $\bb{A}$ obtained via \cref{algo:hhQR} with mixed precision FLOPs where inner products are computed in precision $h$ then cast down.
	All other operations are carried out in precision $l$.
	Then,
	\begin{align}
%	\hat{\bb{R}} &= \bb{R} + \Delta \bb{R}__{mpHQR} = \fl(\hat{\bb{P}}_n\cdots\hat{\bb{P}}_1 \bb{A}),\\
%	\hat{\bb{Q}} &= \bb{Q} + \Delta \bb{Q} = \fl(\hat{\bb{P}}_1\cdots\hat{\bb{P}}_n \bb{I}),\\
	\|\Delta \bb{R}_{mpHQR2}[:,j]\|_2&\leq (\tilde{\gamma}_n^{(l)}+n\tilde{\gamma}_m^{(h)}) \|\bb{A}[:,j]\|_2,\;\; \|\Delta \bb{R}_{mpHQR2}\|_F\leq (\tilde{\gamma}_n^{(l)}+n\tilde{\gamma}_m^{(h)}) \|\bb{A}\|_F \label{eqn:mpHQR2R}\\
	\|\Delta \bb{Q}[:,j]_{mpHQR2}\|_2&\leq (\tilde{\gamma}_n^{(l)}+n\tilde{\gamma}_m^{(h)}),\;\; \|\Delta \bb{Q}_{mpHQR2}\|_F \leq n^{1/2} (\tilde{\gamma}_n^{(l)}+n\tilde{\gamma}_m^{(h)})\label{eqn:mpHQR2Q}.
	\end{align}
%	Let $\bb{A}+\Delta \bb{A} = \hat{\bb{Q}}\hat{\bb{R}}$, where $\hat{\bb{Q}}$ and $\hat{\bb{R}}$ are obtained via Algorithm~\ref{algo:hhQR}.
%	Then the backward e|rror is
%	\begin{equation}
%	\|\Delta \bb{A}\|_F \leq n^{3/2}\tilde{\gamma}_{m}\|\bb{A}\|_F.
%	\end{equation}
\end{theorem}
Unsurprisingly, the inner product mixed precision setting yields higher error bounds as it uses more low precision arithmetic than the settings described in \cref{sec:mp-3}. 
In the next sections we analyze using {\tt mpHQR2} instead of {\tt HQR} within \cref{algo:blockHQR,algo:par_tsqr}.
%TODO: what do these bounds mean?

\subsubsection{BQR round down at inner product: {\tt mpBQR2}}
Now, we analyze \cref{algo:blockHQR} implemented with \cref{assump:mp}. 
At the $k^{th}$ block, we first apply the mixed precision HQR summarized in \cref{thm:mpHQR}.
Next, we construct the WY representation, where we can now use \cref{eqn:mpdelty,eqn:mpapplyP,lem:3.7} to form
\begin{equation}
	\|\hat{\bb{X}}_{k}^{(l)}- \bb{X}_k\|_F = \|(\hat{\bb{P}}_k^{(1)}\cdots \hat{\bb{P}}_k^{(r)})-(\bb{P}_k^{(1)}\cdots \bb{P}_k^{(r)}))\|_F \leq \tilde{\gamma}_{r}^{(l)} + r\tilde{\gamma}_{m}^{(h)}.
\end{equation}
Then, the 2-norm bound for the $j^{th}$ column of the $\bb{R}$ factor and the Frobenius norm bound for the orthogonal factor resulting from {\tt mpBQR2} are
\begin{align}
	\|\hat{\bb{R}}_{mpBQR2}[:,j]\|_2 &= \|\hat{\bb{X}}_1\cdots\hat{\bb{X}}_N\bb{A}[:,j]\|_2\leq\left( N\tilde{\gamma}_{r}^{(l)} + n\tilde{\gamma}_{m}^{(h)}\right)\|\bb{A}[:,j]\|_2,\\
%\end{equation}
%and the Frobenius norm error bound for the orthogonal factor is, 
%\begin{equation}
	\|\hat{\bb{Q}}_{mpBQR2}\|_F &\leq n^{1/2}\left(N\tilde{\gamma}_{r}^{(l)} + n\tilde{\gamma}_{m}^{(h)}\right) \approx \left(1+\frac{M_{l,h}}{m}\right)n^{3/2}\tilde{\gamma}_{m}^{(h)}. \label{eqn:mpBQR2}
\end{align}
Note that this error bound is of the same order as the error bound for {\tt mpHQR2}, shown in \cref{eqn:mpHQR2Q}.
The corresponding error bound for {\tt mpBQR3} of \cref{sec:mp-3b} yielded low precision errors $r$ times smaller than that from 
%a multiplicative factor of $(1+\frac{M_{l,h}}{rm})$ (see \cref{eqn:mpBQR3}).
%This implies that 
using \cref{assump:mp} inner products,
% introduces low precision error $r$ times larger than the low precision errors incurred from {\tt mpBQR3} (\cref{algo:mpBQR}),
 an unsurprising result as intermediate results are cast down more often in {\tt mpBQR2}.
Furthermore, the $\tilde{\gamma}^{(l)}$ in this section requires $c=\cO(25)$, whereas the same notation in \cref{sec:mp-3b} assumes $c$ to be a \emph{small} positive integer.
Therefore, the numerical stability of {\tt mpBQR2} is guaranteed at smaller matrix sizes than the numerical stability of {\tt mpBQR3} and BQR in high precision.
While it is technically possible that the low precision errors introduced from utilizing \cref{assump:mp} do not dominate the errors incurred in {\tt mpBQR2} and {\tt mpHQR2} when $m\gg M_{l,h}$ and can result in accuracy comparable to that of {\tt mpBQR3} and high precision BQR, our numerical results in \cref{sec:NE} show that {\tt mpHQR2} is already unstable at $m\approx M_{l,h}$.
%the mixed precision inner product can still had non-leading order error terms to the worst-case scenario.
\subsubsection{TSQR round down at inner product: {\tt mpTSQR2}}
Finally, we consider using \cref{assump:mp} in \cref{algo:par_tsqr}.
This corresponds to replacing every instance of $n\tilde{\gamma}_{m'}$ for $m'\in\{2n, m2^{-L}\}$ in \cref{thm:moriTSQR} with $\tilde{\gamma}_n^{(l)} + n\tilde{\gamma}_{m'}^{(h)}$.
We first consider the norm errors for the $j^{th}$ column of the $\bb{Q}$ factor computed by this mixed precision variant of \cref{algo:par_tsqr},
\begin{equation}
	\|\hat{\bb{Q}}_{mpTSQR2}[:,j] -\bb{Q}[:,j]\|_2 \leq (L+1)\tilde{\gamma}_n^{(l)} +n(\tilde{\gamma}_{m2^{-L}}^{(h)} + L\tilde{\gamma}_{ 2n}^{(h)}).\label{eqn:mptsqr2Qcol}
\end{equation} 
Then, the matrix norm error bound is 
\begin{align}
\|\hat{\bb{Q}}_{mpTSQR2}-\bb{Q}\|_F \leq n^{1/2}(L+1)\tilde{\gamma}_n^{(l)} +n^{3/2}(\tilde{\gamma}_{m2^{-L}}^{(h)} + L\tilde{\gamma}_{ 2n}^{(h)})\\
\approx \left(1+ \frac{M_{l,h}L}{m2^{-L}+ 2Ln}\right)n^{3/2}(\tilde{\gamma}_{m2^{-L}}^{(h)} + L\tilde{\gamma}_{ 2n}^{(h)}),\label{eqn:mptsqr2Q}
\end{align}
and contributes larger low precision rounding errors than in \cref{eqn:mpTSQR3}.
%\paragraph{{\tt mpTSQR2} and {\tt mpHQR2} error bound comparison}
If the {\tt mpTSQR2} error bound were to outperform that of {\tt mpHQR2}, we now need integers $m, n > 0$, and $L\geq 0$ that satisfy
\begin{equation*}
1\gg n^{1/2}\left(\tilde{\gamma}_{n}^{(l)} + n\tilde{\gamma}_{m}^{(h)}\right) \gg n^{1/2}\left((L+1)\tilde{\gamma}_n^{(l)} +n(\tilde{\gamma}_{m2^{-L}}^{(h)} + L\tilde{\gamma}_{ 2n}^{(h)})\right).%,
\end{equation*}
%where $d=\lfloor\frac{(m-1) u_s}{u_w}\rfloor$, $d_1 = \lfloor{(\frac{m}{2^L}-1)\frac{u_s}{u_w}\rfloor}$, and $d_2 =\lfloor \frac{(2n-1)u_s}{u_w}\rfloor$. 
In contrast to the analysis for uniform precision settings, large $L$ values do not necessarily reduce the error bounds of TSQR. 
While large $L$ can imply $m\gg m2^{-L}+2Ln$, it does not always lead to $d \gg d_1+Ld_2$.
Although the theoretical error bounds do not give a clear indication of the worst-case performances of HQR and TSQR in mixed-precision settings, TSQR outperformed HQR on ill-conditioned matrices within our numerical simulations.
These experiments are discussed in detail in the next section.%Section~\ref{sec:NE}.
%TODO: conclude