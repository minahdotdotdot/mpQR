We introduce the Householder QR factorization algorithm (HQR) in \cref{sec:HQR} and two block variants that use HQR within the block in \cref{sec:BQR,sec:TSQR}. 
The blocked HQR (BQR) in \cref{sec:BQR} partitions the columns of the target matrix and utilizes mainly level-3 BLAS operations and is a well-known algorithm that uses the WY representation of \cite{Bischof1987}.
%\cite{Schreiber1989}.
In contrast, the Tall-and-Skinny QR (TSQR) in \cref{sec:TSQR} partitions rows of the matrix and takes a communication-avoiding divide-and-conquer approach that can be easily parallelized (see \cite{Demmel2007}).
\subsection{Householder QR (HQR)}\label{sec:HQR}
The HQR algorithm uses Householder transformations to zero out elements below the diagonal of a matrix (see \cite{Householder1958}). 
We present this as zeroing out all but the first element of some vector, $\bb{x}\in\R^m$.
\begin{lemma}
	Given vector $\bb{x}\in\R^{m}$, there exist Householder vector, $\bb{v}$, and Householder transformation matrix, $\bb{P}_{\bb{v}}$, such that $\bb{P}_{\bb{v}}$ zeros out $\bb{x}$ below the first element. 
	\begin{equation}
	\begin{alignedat}{3} 
	\sigma =& -\rm{sign}(\bb{x}_1)\|\bb{x}\|_2, &&\quad  \bb{v} = \bb{x} -\sigma \hat{e_1},\\
	\beta = & \frac{2}{\bb{v}^{\top}\bb{v}}=-\frac{1}{\sigma\bb{v}_1}, && \quad \bb{P}_{\bb{v}}=  \bb{I}_{m} - \beta \bb{v}\bb{v}^{\top}.
	\end{alignedat}
	\label{eqn:HH} 
	\end{equation}
	The transformed vector, $\bb{P_vx}$, has the same 2-norm as $\bb{x}$ since Householder transformations are orthogonal: $\bb{P}_{\bb{v}}\bb{x} = \sigma\hat{\bb{e}_1}$.
	In addition, $\bb{P}_{\bb{v}}$ is symmetric and orthogonal, $\bb{P}_{\bb{v}}=\bb{P}_{\bb{v}}^{\top}=\bb{P}_{\bb{v}}^{-1}$.
	\label{lem:hhvec}
\end{lemma}
Given $\bb{A}\in\R^{m\times n}$ and Lemma \ref{lem:hhvec}, HQR is done by repeating the following processes until only an upper triangle matrix remains.
For $i = 1, 2, \cdots, n,$
\begin{enumerate}[Step 1)]
	\item Compute $\bb{v}$ and $\beta$ that zeros out the $i^{th}$ column of $\bb{A}$ beneath $a_{ii}$ (see \cref{algo:hh_v2}), and
	\item Apply $\bb{P}_{\bb{v}}$ to the bottom right partition, $\bb{A}[i:m, i:n]$ (lines 4-6 of \cref{algo:hhQR}).
\end{enumerate}

Consider the following $4$-by-$3$ matrix example adapted from \cite{Higham2002}. 
Let $\bb{P_i}$ represent the $i^{th}$ Householder transformation of this algorithm. 
\[\bb{A} = \left[ \begin{array}{ccc}
\times & \times & \times \\
\times & \times & \times \\
\times & \times & \times \\
\times & \times & \times
\end{array}
\right]\xrightarrow{\text{apply $\bb{P_1}$ to $\bb{A}$}}\left[ \begin{array}{c|cc}
\times & \times & \times \\ \hline
0 & \times & \times \\
0 & \times & \times \\
0 & \times & \times
\end{array}
\right]
\xrightarrow{\text{apply $\bb{P_2}$ to $\bb{P_1}\bb{A}$}}\]
\[ \left[
\begin{array}{cc|c}
\times & \times & \times \\
0 & \times & \times \\ \hline
0 & 0 & \times \\
0 & 0 & \times 
\end{array} \right]
\xrightarrow{\text{apply $\bb{P_3}$ to $\bb{P_2}\bb{P_1}\bb{A}$}} \left[ \begin{array}{ccc}
\times & \times & \times \\
0 & \times & \times \\
0 & 0 & \times \\
0 & 0 & 0 
\end{array}\right] = \bb{P_3}\bb{P_2}\bb{P_1}\bb{A}=:\bb{R} \] 
Then, the $\bb{Q}$ factor for a full QR factorization is $\bb{Q}:=\bb{P_1}\bb{P_2}\bb{P_3}$ since $\bb{P_i}$'s are symmetric, and the thin factors for a general matrix $\bb{A}\in\R^{m\times n}$ are
\begin{equation}
\bb{Q}_{\text{thin}} = \bb{P_1} \cdots \bb{P_n}\bb{I}_{m\times n}\quad \text{and} \quad \bb{R}_{\text{thin}} = \bb{I}_{m\times n}^{\top}\bb{P_n}\cdots \bb{P_1}\bb{A}.
\end{equation}

\begin{algorithm2e}[H]
	\DontPrintSemicolon % Some LaTeX compilers require you to use \dontprintsemicolon instead
	\KwIn{$\bb{x}\in\R^m$}
	\KwOut{$\bb{v}\in\R^m$, and $\sigma, \beta\in\R$ such that $(I-\beta \bb{v}\bb{v}^{\top})\bb{x} = \pm \|\bb{x}\|_2 \hat{e_1} = \sigma\hat{e_1}$ }
	%\tcc{We choose the sign of sigma to avoid cancellation of $\bb{x}_1$ (As is the standard in LAPACK, LINPACK packages \cite{Higham2002}). This makes $\beta>0$.}
	$\bb{v}\gets$ {\tt copy}($\bb{x}$)\\
	$\sigma \gets -\rm{sign}(\bb{x}_1)\|\bb{x}\|_2$\\
	$\bb{v}_1 \gets \bb{x}_1-\sigma$ \\
	%\tcp*{This is referred to as $\bb{\tilde{v}}_1$ later on.} 
	$\beta \gets -\frac{\bb{v}_1}{\sigma}$\\
	%$\bb{v} \gets \frac{1}{\bb{v}_1}\bb{v}$\\
	\Return $\beta$, $\bb{v}/\bb{v}_1$, $\sigma$
	\caption{$\beta$, $\bb{v}$, $\sigma = {\tt hh\_vec}(\bb{x})$. Given a vector $\bb{x}\in\R^n$, return $\bb{v}$, $\beta$, $\sigma$ that satisfy $(I-\beta \bb{v}\bb{v}^{\top})\bb{x} =\sigma\hat{e_1}$ and $\bb{v}_1=1$ (see \cite{LAPACK, Higham2002}).}
	\label{algo:hh_v2}
\end{algorithm2e}

\begin{algorithm2e}
	\DontPrintSemicolon
	\KwIn{$A\in\R^{m \times n}$ where $m \geq n$.}
	
	\KwOut{$\bb{V}$,$\bm{\beta}$, $\bb{R}$}
	%	\tcc{$\bb{v}_i = V[i:m, i] \in \R^{m-(i-1)}$ and $\bb{B}_i = \bb{B}[i:m, i:d] \in \R^{(m-(i-1))\times(d-(i-1))}$.}
	$\bb{V}, \bm{\beta} \gets \bb{0}_{m\times n}, \bb{0}_m$ \\
	
	\For{$i=1 : n$}{
		$\bb{v}, \beta, \sigma \gets \mathrm{hh\_vec}(\bb{A}[i:\mathrm{end}, i])$\\	
		$\bb{V}[i:\mathrm{end},i]$, $\bm{\beta}_i$,  $\bb{A}[i,i] \gets \bb{v}, \beta, \sigma$\\
		%\tcp*{Stores the Householder vectors and constants.}
		%\tcc{The next two steps update $\bb{A}$.}
		$\bb{A}[i+1:\mathrm{end}, i]\gets \mathrm{zeros}(m-i)$\\
		$\bb{A}[i:\mathrm{end}, i+1:\mathrm{end}]\gets \bb{A}[i:\mathrm{end}, i+1:\mathrm{end}] - \beta \bb{v} \bb{v}^{\top}\bb{A}[i:\mathrm{end}, i+1:\mathrm{end}]$
		
	}
	\Return $\bb{V}$, $\bm{\beta}$, $\bb{A}[1:n, 1:n]$
	\caption{$\bb{V}$, $\bm{\beta}$, $\bb{R}$ = ${\tt householder\_qr}(A)$. Given a matrix $\bb{A}\in\R^{m\times n}$ where $m\geq n$, return matrix $\bb{V}\in\R^{m\times n}$, vector $\bm{\beta}\in\R^{n}$, and upper triangular matrix $\bb{R}$. An orthogonal matrix $\bb{Q}$ can be generated from $\bb{V}$ and $\bm{\beta}$, and $\bb{QR}=\bb{A}$.}
	\label{algo:hhQR}
\end{algorithm2e}

\subsection{Block HQR with partitioned columns}\label{sec:BQR}
\subsection{Block HQR with partitioned rows : Tall-and-Skinny QR (TSQR)}\label{sec:TSQR}