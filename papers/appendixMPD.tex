\section{Proofs}\label{appendix:A}
\subsection{Lemma \ref{lem:mp} (Equation \ref{lem:mp1})}
\begin{proof}
	We wish to round up to the lower precision, $p$, since $1\gg u_p \gg u_s$.  
	Recall that $d := \left\lfloor k_s u_s  / u_p \right\rfloor$ and $r \leq \left\lfloor u_p  / u_s \right\rfloor$,
	and note
	%%% \begin{equation*}
	$ k_pu_p+k_su_s = (k_p+d)u_p + r u_s \leq (k_p+d+1)u_p$. Then,
	%%% \end{equation*}
	\begin{align*}
	\gamma_{p}^{(k_p)}+\gamma_{s}^{(k_s)}+\gamma_{p}^{(k_p)}\gamma_{s}^{(k_s)} 
&= \frac{k_pu_p}{1-k_pu_p} + \frac{k_su_s}{1-k_su_s} + \frac{k_pu_p}{1-k_pu_p}\frac{k_su_s}{1-k_su_s} \\
	= \frac{k_pu_p+k_su_s-k_pk_su_pu_s}{1-(k_pu_p+k_su_s)+k_pk_su_pu_s} %%% \\
	&\leq \frac{(k_p+d+1)u_p-k_pk_su_pu_s}{1-(k_p+d+1)u_p+k_pk_su_pu_s} \\
	&< \frac{(k_p+d+1)u_p}{1-(k_p+d+1)u_p} = \gamma_{p}^{(k_p+d+1)}
	\end{align*}
\end{proof}

\subsection{Inner Products}
\label{appendix:IP}
\subsubsection{Lemma \ref{lem:ip_a}}
Let $\dd_p$ and $\dd_s$ be rounding error incurred from products and summations.
They are bounded by $|\dd_p| < u_p$ and $|\dd_s| < u_s$, following the notation in \cite{Higham2002}. Let $s_k$ denote the $k^{th}$ partial sum, and let $\hat{s_k}$ denote the floating point representation of the calculated $s_k$.
Then,
\begin{align*}
	\hat{s_1} &= \fl (\bb{x}_1\bb{y}_1) = \bb{x}_1\bb{y}_1(1 + \dd_{p,1}),\\
	\hat{s_2} &= \fl(\hat{s_1} + \bb{x}_2\bb{y}_2), \\
	&= \left[\bb{x}_1\bb{y}_1(1 + \dd_{p,1}) + \bb{x}_2\bb{y}_2(1 + \dd_{p,2})\right](1+\dd_{s,1}),\\
	\hat{s_3} &= \fl(\hat{s_2}+\bb{x}_3\bb{y}_3), \\
	&= \left(\left[\bb{x}_1\bb{y}_1(1 + \dd_{p,1}) + \bb{x}_2\bb{y}_2(1 + \dd_{p,2})\right](1+\dd_{s,1})  + \bb{x}_3\bb{y}_3(1+\dd_{p,3})\right)(1+\dd_{s,2}).
\end{align*}
We can see a pattern emerging. 
The error for a general length $m$ vector dot product is then:
\begin{equation}
\label{eqn:dperr_1}
\hat{s_m} = (\bb{x}_1\bb{y}_1+\bb{x}_2\bb{y}_2)(1+ \dd_{p,1})\prod_{k=1}^{m-1}(1+\dd_{s,k}) + \sum_{i=3}^n \bb{x}_i\bb{y}_i(1+\dd_{p,i})\left(\prod_{k=i-1}^{m-1}(1+\dd_{s,k})\right),
\end{equation}
where each occurrence of $\dd_p$ and $\dd_s$ are distinct, but are still bound by $u_p$ and $u_s$.

Using Lemma \ref{lem:gamma}, we further simplify:
\begin{align*}
\fl(\bb{x}^{\top}\bb{y}) &= \hat{s_m} = (1+\tth_p^{(1)})(1+\tth_s^{(m-1)})\bb{x}^{\top}\bb{y}%%%\\
%%%&
= (\bb{x}+\Delta\bb{x})^{\top}\bb{y} = \bb{x}^{\top}(\bb{y}+\Delta\bb{y})
\end{align*}
Here $\Delta\bb{x}$ and $\Delta\bb{y}$ are vector perturbations.

By using Lemma \ref{lem:mp}, Equation \ref{lem:mp1}, we can bound the perturbations componentwise.
Let $d:=\lfloor\frac{(m-1)u_s}{u_p}\rfloor$ such that $(m-1)u_s = d u_p + r u_s$. Then,
\begin{align*}
|\Delta \bb{x}| &\leq \gamma_p^{(d+2)}|\bb{x}| %%% \\
\qquad \mbox{and} \qquad
|\Delta \bb{y}| %%%&
\leq \gamma_p^{(d+2)}|\bb{y}|.
\end{align*}
Furthermore, these bounds lead to a forward error result as shown in Equation \ref{eqn:ipforward},
\begin{equation}
\label{eqn:ipforward}
|\bb{x}^{\top}\bb{y}-\fl(\bb{x}^{\top}\bb{y})| \leq \gamma_p^{(d+2)}|\bb{x}|^{\top}|\bb{y}|.
\end{equation}
%
%While this result does not guarantee a high relative accuracy when $|\bb{x}^{\top}\bb{y}| \ll |\bb{x}|^{\top}|\bb{y}|$, high relative accuracy is expected in some special cases.
%For example, let $\bb{x}=\bb{y}$.
%Then we have exactly $|\bb{x}^{\top}\bb{x}| = |\bb{x}|^{\top}|\bb{x}|=\|\bb{x}\|_2^2$.
%This leads to
%\begin{equation}
%\left|\frac{\|\bb{x}\|_2^2 - \fl(\|\bb{x}\|_2^2)}{\|\bb{x}\|_2^2}\right| \leq \gamma_p^{(d+2)}
%\end{equation}
%

\subsubsection{Corollary \ref{lem:ip_b}}
This proof follows similarly to the proof for Lemma \ref{lem:ip_a}.
Since no error is incurred in the multiplication portion of the inner products, $\dd_s$ and $\dd_{st}$ are rounding error incurred from summations and storage.
The error for a general $m$-length vector dot product is then
\begin{equation}
\label{eqn:dperr_2}
\hat{s_m} = (\bb{x}_1\bb{y}_1+\bb{x}_2\bb{y}_2)\prod_{k=1}^{m-1}(1+\dd_{s,k}) + \sum_{i=3}^n \bb{x}_i\bb{y}_i\left(\prod_{k=i-1}^{m-1}(1+\dd_{s,k})\right).
\end{equation}
Using Lemma \ref{lem:gamma}, we further simplify, 
\begin{equation*}
\fl(\bb{x}^{\top}\bb{y}) = \hat{s_m} = (1+\tth_s^{(m-1)})\bb{x}^{\top}\bb{y}= (\bb{x}+\Delta\bb{x})^{\top}\bb{y} = \bb{x}^{\top}(\bb{y}+\Delta\bb{y}).
\end{equation*}
Here $\Delta\bb{x}$ and $\Delta\bb{y}$ are vector perturbations.

By using Lemma \ref{lem:mp} equation \ref{lem:mp1}, we can bound the perturbations componentwise.
Let $d:=\lfloor\frac{(m-1)u_s}{u_p}\rfloor$ such that $(m-1)u_s = d u_p + r u_s$. 
\begin{align*}
|\Delta \bb{x}| &\leq \gamma_p^{(d+1)}|\bb{x}| %%%\\
\qquad \mbox{and} \qquad
|\Delta \bb{y}| %%%&
\leq \gamma_p^{(d+1)}|\bb{y}| 
\end{align*}
Furthermore, these bounds lead to a forward error result as shown in Equation \ref{eqn:ipforward2},
\begin{equation}
\label{eqn:ipforward2}
|\bb{x}^{\top}\bb{y}-\fl(\bb{x}^{\top}\bb{y})| \leq \gamma_p^{(d+1)}|\bb{x}|^{\top}|\bb{y}|.
\end{equation}

\subsection{Proof for Mixed-Precision HQR result}
\label{Appendix:HQR}
Here, we show a few results that are necessary for the proof for Theorem~\ref{thm:feHQR}.
Lemma~\ref{lem:19.2} shows normwise results for a single mixed-precision Householder transformation performed on a vector, and Lemma~\ref{lem:19.3} builds on Lemma~\ref{lem:19.2} to show normwise results for multiple mixed-precision Householder transformations on a vector. 
We build column-wise results for HQR based on these lemmas and then compute the matrix norms at the end.
\begin{lemma}
	\label{lem:19.2}
	Let $\bb{x}\in\R^m$ and consider the computation of $\bb{y}=\hat{\bb P_v}\bb{x} = \bb{x}-\hat{\beta}\hat{\bb{v}}\hat{\bb{v}}^{\top}\bb{x}$, where $\hat{\bb{v}}$ has accumulated error shown in Lemma~\ref{lem:HQRv}.
	Then, the computed $\hat{\bb{y}}$ satisfies 
	\begin{equation}
	\hat{\bb y} = (\bb{P}+\bb{\Delta P}) \bb{x},\quad \|\bb{\Delta P}\|_F\leq\gamma_w^{(6d+6z+13)},
	\end{equation}
	where $\bb{P} = \bb{I}-\beta\bb{v}\bb{v}^{\top}$ is a Householder transformation.
\end{lemma}
\begin{proof}
	Recall that the computed $\hat{\bb y}$ accumulates component-wise error shown in Equation~\ref{eqn:applyP}.
	Even though we do not explicitly form $\bb{P}$, forming the normwise error bound for this matrix makes the analysis simple.
	First, recall that any matrix $\bb{A}$ with rank $r$ has the following relations between its 2-norm and Frobenius norm,
	%%%\begin{equation}
	$\|\bb{A}\|_2\leq\|\bb{A}\|_F\leq\sqrt{r}\|\bb{A}\|_2$.
	%%%\end{equation}
	Then, we have 
	\begin{equation}
	\label{eqn:19.2a}
	\|\bb{y}\|_2 = \|\bb{P x}\|_2 \leq \|\bb{P}\|_2 \|\bb{x}\|_2 = \|\bb{x}\|_2,
	\end{equation}
	since $\bb{P}$ is orthogonal and $\|\bb{P}\|_2=1$.
	We now transition from the componentwise error to normwise error for $\bb{\Delta y}$, and write $\tilde{z} = 6d+6z+13$:
	\begin{equation}
	\label{eqn:19.2b}
	\|\bb{\Delta y}\|_2 = \left(\sum_{i=1}^m \bb{\Delta y}_i^2\right)^{1/2} \leq \gamma_w^{(\tilde{z})}\left(\sum_{i=1}^m \bb{y}_i^2\right)^{1/2} =  \gamma_w^{(\tilde{z})}\|\bb{y}\|_2
	\end{equation}
	Combining Equations~\ref{eqn:19.2a} and \ref{eqn:19.2b}, we find
	\begin{equation}
	\frac{\|\bb{\Delta y}\|_2}{\|\bb{x}\|_2} \leq \gamma_w^{(\tilde{z})}. \label{eqn:19.2c}
	\end{equation}
	Now, notice that $\bb{\Delta P}$ is exactly $\frac{1}{\bb{x}^{\top}\bb{x}}\bb{\Delta y}\bb{x}^{\top}$; thus, 
	\begin{align*}
	(\bb{P}+\bb{\Delta P}) \bb{x} &= (\bb{P}+\frac{1}{\bb{x}^{\top}\bb{x}}\bb{\Delta y}\bb{x}^{\top})\bb{x}    %%%\\
	%%%&
=\bb{P}\bb{x}  + \frac{\bb{x}^{\top}\bb{x}}{\bb{x}^{\top}\bb{x}}\bb{\Delta y} = \bb{y} + \bb{\Delta y}
	\end{align*}
	We can compute the Frobenius norm of $\bb{\Delta P}$ by using $\bb{\Delta P}_{ij} = \frac{1}{\|\bb{x}\|_2^2}\bb{\Delta y}_i\bb{x}_j$.
	\begin{align*}
	\|\bb{\Delta P}\|_F %%&= \left(\sum_{i=1}^m\sum_{j=1}^m\Delta \bb{P}_{ij}^2\right)^{1/2} 
	= \left(\sum_{i=1}^m\sum_{j=1}^m\left(\frac{1}{\|\bb{x}\|_2^2}\bb{\Delta y}_i\bb{x}_j\right)^2\right)^{1/2} %%%\\
	%%%&= \left(\frac{1}{\|\bb{x}\|_2^4}\sum_{i=1}^m\sum_{j=1}^m\bb{\Delta y}_i^2\bb{x}_j^2\right)^{1/2} 
	%%%=\frac{1}{\|\bb{x}\|_2^2}\left(\sum_{i=1}^m \bb{\Delta y}_i^2\left( \sum_{j=1}^m\bb{x}_j^2\right)\right)^{1/2}\\
	%%%&=\frac{1}{\|\bb{x}\|_2^2}\left(\|\bb{x}\|_2^2\sum_{i=1}^m\bb{\Delta y}_i^2 \right)^{1/2}
	%%%=  \frac{\|\bb{x}\|_2\|\bb{\Delta y}\|_2}{\|\bb{x}\|_2^2} 
        =  \frac{\|\bb{\Delta y}\|_2}{\|\bb{x}\|_2}
	\end{align*}
	Finally, using Equation~\ref{eqn:19.2c}, we find $\|\bb{\Delta P}\|_F \leq \gamma_w^{(\tilde{z})}$.
\end{proof}

%\begin{lemma}
%	\label{lem:3.7}
%	If $\bb{P}_j + \bb{\Delta P}_j\in\R^{m\times m}$ satisfies $\|\bb{\Delta P}_j\|_F\leq\delta_j\|\bb{P}_j\|_2$ for all $j$, then
%	\begin{equation}
%	\left|\left|\prod_{j=0}^m \left(\bb{P}_j + \bb{\Delta P}_j\right) - \prod_{j=0}^m \bb{P}_j\right|\right|_F\leq \left(\prod_{j=0}^m(1+\delta_j)-1\right)\prod_{j=0}^m\|\bb{P}_j\|_2
%	\end{equation}
%\end{lemma}

\begin{lemma}
	\label{lem:19.3}
	Consider applying a sequence of transformations in the set $\{\bb{P}_j\}_{j=1}^r\subset\R^{m\times m}$ to $\bb{x}\in\R^m$, where $\bb{P}_j$'s are all Householder transformations and where we will assume that $r\gamma_w^{(\tilde{z})}<\frac{1}{2}.$ 
	Let $\bb{y} = \bb{P}_r\bb{P}_{r-1}\cdots\bb{P}_1\bb{x} = \bb{Q}^{\top}\bb{x}$.
	Then, $\hat{\bb{y}} = (\bb{Q}+\bb{\Delta Q})^{\top}\bb{x}$, where 
	\begin{equation}
	\|\bb{\Delta Q}\|_F \leq r\gamma_w^{(\tilde{z})},\quad  \|\bb{\Delta y}\|_2 \leq r \gamma_w^{(\tilde{z})} \|\bb{y}\|_2.\label{eqn:19.3}
	\end{equation}
	In addition, if we let $\hat{\bb{y}} =\bb{Q}^{\top}(\bb{x} + \bb{\Delta x})$, then 
	\begin{equation}
	\|\bb{\Delta x}\|_2 \leq r \gamma_w^{(\tilde{z})} \|\bb{x}\|_2.\label{eqn:19.3c}
	\end{equation}
\end{lemma}

\begin{proof}
	As was for the proof for Lemma~\ref{lem:19.2}, we know $\bb{\Delta Q}^{\top} = \frac{1}{\|\bb{x}\|_2^2}\bb{\Delta y}\bb{x}^{\top}$.
	Recall that the HQR factorization applies a series of Householder transformations on $\bb{A}$ to form $\bb{R}$, and applies the same series of Householder transformations in reverse order to $\bb{I}$ to form $\bb{Q}$.
	Therefore, it is appropriate to assume that $\bb{x}$ is exact in this proof, and we form a forward bound on $\hat{\bb{y}}$.
	However, we can still easily switch between forward and backward errors in the following way:
	\begin{align*}
	\hat{\bb{y}} &= \bb{y} + \bb{\Delta y} = \bb{Q}^{\top} (\bb{x}+\bb{\Delta x}) = (\bb{Q}+\bb{\Delta Q})^{\top} \bb{x},\\
	\bb{\Delta y} &= \bb{\Delta Q}^{\top} \bb{x} = \bb{Q}^{\top}\bb{\Delta x}.
	\end{align*}
	In addition, we can switch between $\bb{\Delta y}$ and $\bb{\Delta x}$ by using the fact that $\|\bb{Q}\|_2 = 1$.
	%So another way of formulating $\bb{\Delta Q}$ is given by $\frac{1}{\|\bb{x}\|_2^2}(\bb{Q}\bb{\Delta x})\bb{x}^{\top}$, where $\bb{\Delta x}$ can be understood as the backward error.\par	
	\paragraph{Error bound for $\|\bb{\Delta y}\|_2$}
	We will first find $\|\bb{\Delta Q}\|_2$, where this is NOT the forward error from forming $\bb{Q}$ with Householder transformations, but rather a backward error in accumulating Householder transformations. 
	From Lemma~\ref{lem:19.2}, we have$ \|\bb{\Delta P}\|_F\leq\gamma_w^{(\tilde{z})} = \gamma_w^{(\tilde{z})} \|\bb{P}\|_2$ for any Householder transformation $\bb{P}\in\R^{m\times m}$, where $\tilde{z} = 6d+6z+13$ and $d=\lfloor\frac{(m-1)u_s}{u_w}\rfloor$, $z\in\{1,2\}$.
	Therefore, this applies to the sequence of $\bb{P}_i$'s that form $\bb{Q}$ as well.
	
	We will now use Lemma 3.7 from \cite{Higham2002} to bound $\bb{\Delta Q}$:
	\begin{align*}
	\bb{\Delta Q}^{\top}& = \left(\hat{\bb{Q}} - \bb{Q}\right)^{\top}= \prod_{i=r}^{1}\left(\bb{P}_i +\bb{\Delta P}_i\right) - \prod_{i=r}^{1}\bb{P}_i,\\
	 \|\bb{\Delta Q}\|_F = \|\bb{\Delta Q}^{\top}\|_F  &= \left|\left| \prod_{i=r}^{1}\left(\bb{P}_i +\bb{\Delta P}_i\right) - \prod_{i=r}^{1}\bb{P}_i \right|\right|_F,\\
	&\leq \left(\prod_{i=r}^1(1+\gamma_w^{(\tilde{z})})-1\right)\prod_{i=r}^1\|\bb{P}_i\|_2 = \prod_{i=r}^1(1+\gamma_w^{(\tilde{z})})-1.
	\end{align*}
	The last equality results from the orthogonality of Householder matrices.\par
	
	Consider the constant, $(1+\gamma_w^{(\tilde{z})})^r-1$.
	From the very last rule in Lemma~\ref{lem:up}, we can generalize the following:
	\begin{equation*}
	(1+\gamma_w^{(\tilde{z})})^r = (1+\gamma_w^{(\tilde{z})})^{r-2}(1+\gamma_w^{(\tilde{z})})(1+\gamma_w^{(\tilde{z})}) \leq  (1+\gamma_w^{(\tilde{z})})^{r-2}(1+\gamma_w^{(2\tilde{z})}) \leq \cdots \leq (1+\gamma_w^{(r\tilde{z})}).
	\end{equation*}
	So, our quantity of interest can be bound by $(1+\gamma_w^{(\tilde{z})})^r-1 \leq \gamma_w^{(r\tilde{z})}$.
	
	Now we will use the following equivalent algebraic inequalities to get the final result.
	\begin{equation}
	0<a<b<1 \Leftrightarrow 1-a > 1-b \Leftrightarrow \frac{1}{1-a} <\frac{1}{1-b} \Leftrightarrow \frac{a}{1-a} < \frac{b}{1-b}
	\label{eqn:algebra}
	\end{equation}
	In addition, we assume $r\gamma_w^{(\tilde{z})}< \frac{1}{2}$, such that 
	\begin{align*}
	(1+\gamma_w^{(\tilde{z})})^r-1 &\leq \gamma_w^{(r\tilde{z})} = \frac{r\tilde{z}u_w}{1-r\tilde{z}u_w}\qquad\text{(by definition)}\\
	&\leq \frac{r\gamma_w^{(\tilde{z})}}{1-r\gamma_w^{(\tilde{z})}},\text{ since } r\tilde{z}u_w < r\gamma_w^{(\tilde{z})}\qquad\text{(by Equation \ref{eqn:algebra})}\\
	&\leq 2 r \gamma_w^{(\tilde{z})}\qquad\text{(since $r\gamma_w^{(\tilde{z})}< \frac{1}{2}$ implies  $\frac{1}{1-r\gamma_w^{(\tilde{z})}} < 2$)}\\
	&= r\tilde{\gamma}_w^{(\tilde{z})},
	\end{align*}
	where $\tilde{\gamma^{(m)}}:= \frac{cmu}{1-cmu}$ for some small integer, $c$.
	If we had started with $(1+\tilde{\gamma}_w^{(\tilde{z})})^r-1$, we can still find $r\tilde{\gamma}_w^{(\tilde{z})}$ assuming that $2c$ is still a small integer. 
	In conclusion, we have 
	$$
	(1+\gamma_w^{(\tilde{z})})^r-1 \leq r\tilde{\gamma}_w^{(\tilde{z})},
	$$
	which results in the bound for $\bb{\Delta Q}$ as shown in Equation~\ref{eqn:19.3}, $\|\bb{\Delta Q}\|_2 \leq \|\bb{\Delta Q}\|_F \leq r\gamma_w^{(\tilde{z})}$.

	Next, we bound $\|\bb{\Delta y}\|_2 = \|\bb{\Delta Q x}\|_2 \leq \|\bb{\Delta Q}\|_2 \|\bb{x}\|_2 \leq  r\tilde{\gamma}_w^{(\tilde{z})}\|\bb{x}\|_2$.
	
	\paragraph{Bound for $\|\bb{\Delta x}\|_2$}
	We use the above result,
	\begin{equation}
	\|\bb{\Delta x}\|_2 = \|\bb{Q \Delta y}\|_2\| \leq \|\bb{Q}\|_2\|\bb{\Delta y}\|_2 = \|\bb{\Delta y}\|_2 \leq  r\tilde{\gamma}_w^{(\tilde{z})}\|\bb{x}\|_2.
	\end{equation}
	While $r\gamma^{(k)} = r\frac{ku}{1-ku} < \frac{rku}{1-rku} =\gamma^{(rk)}$ holds true when $r>0$ and $rku< 1$ are satisfied, the strict inequality implies that $r\gamma^{(k)}$ is a tighter bound than $\gamma^{(rk)}$.
	However, $\gamma^{(rk)}$ is easier to work with using the rules in Lemma~\ref{lem:gamma}.
\end{proof}

\subsubsection{Proof for Theorem \ref{thm:feHQR}}
First, we use Lemma~\ref{lem:19.3} directly on columns of $\bb{A}$ and $\bb{I}_{m\times n}$ to get a result for columns of $\hat{\bb{R}}$ and $\hat{\bb{Q}}$.
%Figure~\ref{fig:QRerr} shows that each elements of $\hat{\bb{Q}}$ and $\hat{\bb{R}}$ each go through different numbers of Householder transformations and Householder transformations for different lengths of vectors as well. 
We will use the maximum number of transformations and the length of the longest vector on to which we perform a Householder transformation, that is, $n$ transformations of vectors of length $m$. 
%Also, note that $n\gamma^{(k)} = \frac{nku}{1-ku}<\frac{nku}{1-nku} = \gamma^{(nk)}$ as long as $nku <1$.
For $j$ in $\{1, \cdots, n\}$, the $j^{th}$ column of $\bb{R}$ and $\bb{Q}$ are the results of $j$ Householder transformations on $\bb{A}$ and $\bb{I}$:
\begin{align}
\|\bb{\Delta Q}[:,j]\|_2 &\leq j\tilde{\gamma}_w^{(\tilde{z})}\|\hat{e}_{j}\|_2 < \tilde{\gamma}_w^{(j\tilde{z})}, \\
\|\bb{\Delta R}[:,j]\|_2 &\leq j\tilde{\gamma}_w^{(\tilde{z})}\|\bb{A}[:,j]\|_2 < \tilde{\gamma}_w^{(j\tilde{z})}\|\bb{A}[:,j]\|_2.
\end{align}

Finally, we relate columnwise 2-norms to matrix Frobenius norms.
%\begin{lemma}
%	Let $\bb{A}=[\bb{c}_1 \cdots \bb{c}_n]\in\R^{m\times n}$, where $\bb{c_i}\in\R^m$ for $i=1, \cdots, n$. 
%	If $\max_{i \in\{1, \cdots, n\}}\|\bb{c}_i\| = \epsilon$, then
%	\begin{equation}
%	\|\bb{A}\|_F \leq \sqrt{n}\epsilon
%	\end{equation}
%\end{lemma}
%\begin{proof}
%	The Frobenius norm is exactly the 2-norm for vectors. 
%	\begin{equation*}
%	\|\bb{A}\|_F = \left(\sum_{i=1}^n \sum_{j=1}^m \bb{A}_{ij}^2\right)^{1/2} = \left(\sum_{i=1}^n\|\bb{c}_i\|_2^2\right)^{1/2}\leq \left(\sum_{i=1}^n \epsilon^2\right)^{1/2} = \sqrt{n}\epsilon
%	\end{equation*}
%\end{proof}
It is straightforward to see the result for the $\bb{Q}$ factor,
\begin{equation}
\|\bb{\Delta Q}\|_F = \left(\sum_{j=1}^n \|\bb{\Delta Q}[:,j]\|_2^2\right)^{1/2} \leq \left(\sum_{j=1}^n (j\tilde{\gamma}_w^{(\tilde{z})})^2 \|\hat{e}_j\|_2^2\right)^{1/2} \leq n^{3/2}\tilde{\gamma}_w^{(\tilde{z})}.
\end{equation}
Note that we bound $\sum_{j=1}^n j^2$ by $n^3$, but the summation is actually exactly $\frac{n(n+1)(2n+1)}{6}$. 
Therefore, a tighter bound would replace $n^{3/2}$ with $\left(\frac{n(n+1)(2n+1)}{6}\right)^{1/2}$.\par 
We can bound the $\bb{R}$ factor in a similar way,
\begin{equation}
\|\bb{\Delta R}\|_F = \left(\sum_{j=1}^n \|\bb{\Delta R}[:,j]\|_2^2\right)^{1/2} \leq \left(\sum_{j=1}^n (j\tilde{\gamma}_w^{(\tilde{z})})^2 \|\bb{A}[:,j]\|_2^2\right)^{1/2} \leq n\tilde{\gamma}_w^{(\tilde{z})} \|\bb{A}\|_F.
\end{equation}

Obtaining the backward error from the HQR factorization,
$$\bb{\Delta A}=\bb{A}-\hat{\bb{Q}}\hat{\bb{R}}
= \bb{A}-\bb{Q}\hat{\bb{R}} + \bb{Q}\hat{\bb{R}} - \hat{\bb{Q}}\hat{\bb{R}}
= \bb{Q \Delta R} + \bb{\Delta Q} \hat{\bb{R}}.$$
	
A columnwise result for $\hat{\bb{A}}$ is shown by
\begin{align*}
\|\bb{\Delta A}[:,j]\|_2 & = \|(\bb{Q \Delta R} + \bb{\Delta Q} \hat{\bb{R}} )[:,j]\|_2,\\
&\leq \|\bb{Q \Delta R}[:,j]\|_2  + \|\bb{\Delta Q}\hat{\bb{R}}[:,j]\|_2,\\
&\leq \|\bb{\Delta R}[:,j]\|_2 + \|\bb{\Delta Q}\|_2\|\hat{\bb{R}}[:,j]\|_2,\\
&\leq \|\bb{\Delta R}[:,j]\|_2 + \|\bb{\Delta Q}\|_F\|(\bb{R} +\bb{\Delta R})[:,j]\|_2,\\
&\leq j\tilde{\gamma}_w^{(\tilde{z})}\|\bb{A}[:,j]\|_2 + n^{3/2}\tilde{\gamma}_w^{(\tilde{z})} \|(\bb{Q}^{\top}\bb{A} +\bb{\Delta R})[:,j]\|_2,\\
&\leq j\tilde{\gamma}_w^{(\tilde{z})}\|\bb{A}[:,j]\|_2 + n^{3/2}\tilde{\gamma}_w^{(\tilde{z})} \left(\|\bb{A}[:,j]\|_2 +\|\bb{\Delta R}[:,j]\|_2\right),\\
&\leq \left(j\tilde{\gamma}_w^{(\tilde{z})} + n^{3/2}\tilde{\gamma}_w^{(\tilde{z})}  (1+j\tilde{\gamma}_w^{(\tilde{z})})\right)\|\bb{A}[:,j]\|_2,\\
&=n^{3/2}\tilde{\gamma}_w^{(\tilde{z})}  \|\bb{A}[:,j]\|_2,
\end{align*}
where we assume $n\tilde{\gamma}_w^{(\tilde{z})}\ll 1$ and where the last equality sweeps all non-leading order terms into the arbitrary constant $c$ within the definition of $\tilde{\gamma}$,
\begin{equation}
\|\bb{\Delta A}\|_F = \left(\sum_{j=1}^n \|\bb{\Delta A}[:,j]\|_2^2\right)^{1/2} \leq \left(\sum_{j=1}^n (n^{3/2}\tilde{\gamma}_w^{(\tilde{z})})^2 \|\bb{A}[:,j]\|_2^2\right)^{1/2} \leq n^{3/2}\tilde{\gamma}_w^{(\tilde{z})} \|\bb{A}\|_F.
\end{equation}

